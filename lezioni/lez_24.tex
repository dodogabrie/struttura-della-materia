\lez{24}{19-04-2020}{}
\subsection{Regola d'oro di Fermi}%
\label{sub:Regola d'oro di Fermi}

Riprendiamo la teoria delle perturbazioni (anche dipendenti dal tempo), supponiamo di avere una Hamiltoniana per il nostro sistema $H_0$ e di inserire una perturbazione $V(t) $ tale che:
\[
    \overline{V}(t) = \begin{cases}
	&W(t)  \quad  0<t<\tau\\
	&0 	\quad \quad t<0 \text{ e } t>\tau
    \end{cases}
.\] 
L'Hamiltoniana completa sarà la somma di queste due quantità: $H=H_0+\overline{V}(t)$.\\
Cerchiamo le soluzioni alla equazione di Schrodinger dipendente dal tempo:
\[
    i\hbar \frac{\text{d} \psi(t)}{\text{d} t} 
    = \left[H_0+\overline{V}(t) \right]\psi (t) 
.\] 
Sviluppiamo la funzione d'onda sugli autostati di $H_0$ :
\[
    \ket{\psi(t)}=\sum_{n}^{} a_n(t) e^{-i E_n t /\hbar }\ket{\varphi_n}
.\] 
Ipotizziamo inoltre che per tutti i $t<0$ il sistema si trovi in un autostato $\ket{\varphi_m}$ dato. Vogliamo trovare lo stato del sistema all'istante $ \tau$: quello in cui termina la perturbazione.\\
Per $t\le 0$ abbiamo che 
\[\begin{aligned}
    &\ket{\psi_m}(t) = e^{-iE_m t /\hbar } \ket{\varphi_m}\\
    &a_n(t) = \delta_{mn}
.\end{aligned}\]
Al termine della perturbazione mi troverò in uno stato dato dalla sovrapposizione degli altri autostati, tali autostati continueranno ad evolvere (con l'esponenziale della loro energia) e gli $a_n(t)$ per $t\ge \tau$ rimarranno costanti ed uguali a $a(\tau)$.\\
Possiamo chiamare le ampiezze di probabilità $a_n(\tau)$ come $a_{m\to n}(\tau) $ essendo questa l'ampiezza di probabilità dello stato $\varphi_m$ di passare allo stato n-esimo.\\
La probabilità di andare dallo stato $m$ allo stato $n$ sarà data da:
\[
    W_{m\to n}(\tau)=\left|a_{m\to n}(\tau) \right|^2
.\] 
Per trovare tali ampiezze sostituiamo nella equazione di Schrodinger lo sviluppo della $\psi$ e proiettiamo su un generico $\bra{\varphi_n}$.\\
\[
    i\hbar \frac{\text{d} }{\text{d} t} a_{m\to n}(t) 
    = \sum_{l}^{} \bra{n}W(t)\ket{l}e^{i\omega_{n,l} t}a_{m\to l}(t) 
.\] 
Nella quale $\omega_{n,l}$  vale:
\[
\omega_{n,l}=\frac{E_n-E_l}{\hbar }
.\] 
Ci mettiamo nel caso in cui l'elemento di matrice è nullo tra stati uguali $\bra{n}W\ket{n} = 0$ (lo possiamo sempre fare con una base opportuna degli stati) e risolviamo queste equazioni con la condizione 
\[
    a_{m\to n}(t=0) = \delta_{mn}
.\] 
Possiamo riscrivere l'equazione di Schrodinger al primo ordine come:
\[
    i\hbar \frac{\text{d} a_{m\to n}(t) }{\text{d} t} 
    =
    \bra{n}W(t) \ket{m}e^{i\omega_{nm}t }
.\] 
Possiamo integrare questa espressione (mettiamo un pedice per indicare il primo ordine):
\[
    a_{m\to n}^{(1) } (t) =
    \frac{1}{i\hbar }
    \int_{0}^{t} \bra{n}W(t')\ket{m}e^{i\omega_{mn}t'} dt'
.\] 
Agli ordini successivi abbiamo che
\[
    a_{m\to n}^{(2)}(t)=
    \frac{1}{i\hbar }
    \sum_{m}^{} \int_{0}^{t}  
    \bra{k}W(t')\ket{n}e^{i\omega_{kn}t'}a_{m\to n}^{(1)}(t')  dt'
.\] 
Come primo esempio possiamo prendere la $W(t) = V$  indipendente dal tempo. In questo caso abbiamo che:
\[
    a_{m\to n}^{(1)}(t) =
    \frac{1}{i\hbar }\bra{n}W \ket{m} \int_{0}^{t} e^{i\omega_{nm}t'}dt'
.\] 
Quindi possiamo trovare la probabilità di transizione da $m$ a $n$ al tempo $\tau$:
\[
    W^{(1)}_{m\to n}(\tau)=
    \left|a_{m\to n}^{(1)}(\tau) \right|^2 = \frac{2}{\hbar }\left|\bra{n}W \ket{m}\right|^2F(\tau,E_n-E_m) 
.\] 
Questa $F$ è:
\[
    F(\tau,x) = \frac{2}{\left(\frac{x}{\hbar }\right)^2}\sin^2\left(\frac{x\tau}{2\hbar }\right) 
.\] 
Questa funzione ha la seguente forma:
\begin{figure}[H]
    \centering
    \incfig{f-interazione-radiazione-materia}
    \caption{F interazione radiazione materia}
    \label{fig:f-interazione-radiazione-materia}
\end{figure}
In cui 
\[
\left|x\right|=\frac{2\pi\hbar }{\tau}n
.\] 
Maggiore è $\tau$ e più si stringe il massimo della funzione. Se vale
\[
\tau \gg \frac{2\pi\hbar }{\left|x\right|} 
.\] 
Possiamo approssimare tale funzione ad una $\delta$, in tal caso vale la relazione:
\[
    \int_{-\infty}^{\infty} F(x) dx = \tau \hbar \pi 
.\] 
Con questa normalizzazione possiamo scrivere la funzione $F(\tau,x)$: 
\[
    F(\tau,x) =\pi\tau\hbar \delta (x)  
.\] 
Se la perturbazione dura un tempo infinito ($\tau\to \infty$ ) allora l'ultima formula ci dice che il sistema non può cambiare energia: la probabilità non è normalizzabile. \\
Abbiamo trattato il sistema concentrandoci sulla transizione di due livelli $m$ e $n$, in generale possiamo avere una distribuzione di stati continua (come ad esempio la $\rho_e$ nei cristalli ) in tal caso possiamo normalizzare:
\[
    \int\rho(E_n) dE_n = 1
.\] 
Possiamo immaginare che questa densità di stati sia distribuita con un picco largo $\Delta E_n = \hbar \Delta\omega_n $. \\
Possiamo calcolare la probabilità di transizione complessiva come:
\[
    W^{(1)}_{m\to n}(\tau) =
    \frac{2}{\hbar ^2}\int\left|\bra{n}W \ket{m}\right|^2 F(\tau, E_n-E_m)\rho(E_n) dE_n 
.\] 
Tipicamente nei sistemi che trattiamo il primo termine (che è un modulo quadro di un elemento di matrice) non dipende da quali dei singoli stati $m$ o $n$  che andiamo a considerare, quindi può essere tirato fuori dall'integrale.
\[
    W^{(1)}_{m\to n}(\tau) =
    \frac{2}{\hbar ^2}\left|\bra{n}W \ket{m}\right|^2\int\frac{2}{\left(\frac{x}{\hbar }\right)^2}\sin^2\left(\frac{x\tau}{2\hbar }\right) \rho(E_n) dE_n
.\] 
Nell'ipotesi in cui $\tau\gg 2\pi /\Delta\omega$ (nello spazio delle frequenze la $F$ è molto più stretta di $\rho$  ), portiamo allora fuori dall'integrale la $\rho$, ci resta un integrale noto:
\[
    W^{(1)}_{m\to n}(\tau) =
    \frac{2\pi}{\hbar}\left|\bra{\overline{n}}W \ket{m}\right|^2\rho(E_{\overline{n}})\tau 
.\] 
Questa è la regola d'oro di Fermi. Grazie a questa possiamo definire una probabilità $P_{m\to n}$ per unità di tempo dividendo semplicemente per $\tau$.
\[
    \overline{P}_{m\to n}^{(1) } =
    \frac{2\pi}{\hbar}\left|\bra{\overline{n}}W \ket{m}\right|^2\rho(E_{\overline{n}})
.\] 
Possiamo imporre che tale probabilità prima di essere integrata fosse la $\delta$, quindi:
\[
    P_{m\to n}^{(1) } =
    \frac{2\pi}{\hbar}\left|\bra{\overline{n}}W \ket{m}\right|^2\delta (E_n-E_m) 
.\] 
Nel caso opposto invece $\tau\ll\frac{2\pi }{\Delta\omega}$, in tal caso portiamo fuori la $F$ e resta l'integrale della $\rho$. Facendo in questo modo si ottiene 
\[
\overline{P} \propto \tau^4
.\] 
\subsection{Interazione con campo EM}%
\label{sub:Interazione con campo EM}
Abbiamo parlato fin'ora di un potenziale a "impulso quadro", ovvero un potenziale che sale fino ad un $W$  costante e poi torna a 0 al tempo $\tau$. Se interagiamo con il campo elettromagnetico dobbiamo prendere una interazione con un potenziale oscillante, supponiamo di avere una onda piana
\[
V=W^+ e^{-i\omega t} + W^- e^{i\omega t} 
.\] 
Se la perturbazione è di questo tipo possiamo riprendere tutti i passaggi fatti in precedenza e aggiungere all'esponenziale complessa che avevamo prima il pezzo oscillante del campo EM:
\[
    e^{i\omega_{mn}t} \to e^{i(\omega_{mn}-\omega)t }
.\] 
Quindi possiamo utilizzare il caso costante anche per descrivere la situazione oscillante a patto di inserire questo termine.\\
Vediamo il contributo dei due singoli pezzi di $V$:
\[
    a_{m\to n}^{(1)}(t) =
    \frac{1}{i\hbar }\bra{n}W^+ \ket{m} \frac{e^{i(\omega_{nm}-\omega )t}-1}{i\left( \omega_{mn}-\omega\right)}
.\] 
Quindi si ha che:
\[
    W^{(1) }_{m\to n}(\tau) =
    \left|a^{(1)}_{m\to n}(\tau) \right|^2 =
    \frac{2}{\hbar }\left|\bra{n}W^+ \ket{m}\right|^2 F(\tau,E_n-E_m - \hbar \omega) 
.\] 
Nelle ipotesi della regola d'oro di fermi abbiamo che:
\[
    \overline{P}^{(1)}_{m\to n} = \frac{2\pi}{\hbar }\left|\bra{n}W^+ \ket{m}\right|^2\rho(E_n=E_m+\hbar \omega) 
.\] 
Oppure possiamo scrivere senza integrare:
\[
    P^{(1)}_{m\to n} = \frac{2\pi}{\hbar }\left|\bra{n}W^+ \ket{m}\right|^2\delta (E_n-E_m-\hbar \omega) 
.\] 
Prendendo il secondo pezzo di $V$ abbiamo:
\[
    \overline{P}^{(1)}_{m\to n} = \frac{2\pi}{\hbar }\left|\bra{n}W^- \ket{m}\right|^2\rho(E_n=E_m-\hbar \omega) 
.\] 
Quindi i due termini ci descrivono due processi diversi: uno ci fa guadagnare energia e l'altro ce la fa perdere (questo indicano i - e il +). Quindi questi rappresentano il processo di emissione stimolata ed assorbimento, inoltre i due termini $W^+$ e $W^-$ classicamente sono identici. Abbiamo quindi una simmetria completa tra i processi di emissione stimolata ed assorbimento.\\
Spesso non solo gli stati finali hanno una distribuzione iniziale ma ce l'hanno anche gli stati finali, si utilizza quindi spesso la densità di stati ridotta o congiunta:
\[
    \rho_\text{congiunta} (E_n-E_m) 
.\] 
Che ci dice il numero di stati separati da una energia $E_n-E_m$. In questo modo possiamo anche riscrivere la probabilità di transizione:
\[
    \overline{P}^{(1)}_{m\to n} = \frac{2\pi}{\hbar }\left|\bra{n}W^+ \ket{m}\right|^2\rho(E_n-E_m=\hbar \omega) 
.\] 
Dobbiamo adesso capire quando è possibile fermarci allo sviluppo al primo ordine, sarà necessario che:
\[
    W^{(1) }_{m\to n}(\tau) \ll 1
.\] 
Questo significa che:
\[
\frac{1}{\hbar }\left|\bra{n}W^{\pm} \ket{m}\right|^2\tau \ll 1
.\] 
L'elemento di matrice sarà proporzionale al fattore $\v{E}\cdot \v{d}$ quindi sarà necessario considerare tempi di interazione lunghi rispetto alla larghezza di distribuzione dei livelli ma devono essere corti rispetto al reciproco della intensità del campo.\\
In parole povere dobbiamo considerare campi piccoli:
\[
    E^2\tau  \ll 1
.\] 
Vale quindi la seguente disuguaglianza sui tempi di interazione per poter applicare lo sviluppo perturbativo al primo ordine e la regola d'oro di Fermi:
\[
    \frac{2\pi}{\Delta\omega}\ll \tau\ll \frac{1}{\left|\v{E}\right|^2}
.\] 
Abbiamo fin'ora assunto una onda elettromagnetica monocromatica, vedremo cosa succede quando abbiamo invece una distribuzione di campo elettromagnetico (che è quello che succede tipicamente). In generale avremo una densità di energia $u(\nu)d\nu$, in questo caso dovremmo fare la convoluzione della probabilità monocromatica che abbiamo calcolato con tale densità di energia.\\
Prima di far questo vediamo che succede andando agli ordini successivi, prendiamo $a^{(1)}(t) $ e le rimettiamo nella parte destra della equazione differenziale e troviamo com'è fatta la probabilità di transizione. In questo caso senza effettuare i conti vediamo il risultato:
\[
    a^{(2)}_{m\to n}(t) =
    \frac{1}{\hbar ^2} \sum_{n}^{} 
    \frac{\bra{k}W^{\pm}\ket{n} \bra{n}W^{\pm} \ket{m}}{\omega_{nm}\pm \omega}
    \cdot 
    \left\{
	\frac{e^{i(\omega_{km}\pm_2\omega)t}-1}{\omega_{km}\pm_2\omega}
    - \frac{e^{i(\omega_{km}\pm \omega)t}-1}{\omega_{kn}\pm\omega}\right\}
.\] 
La cosa importante è il fatto che abbiamo una somma su tutti gli n. Inoltre la nostra $F$ adesso è piccata nel punto in cui la differenza dei due stati è il doppio di $\omega$ del campo magnetico, quindi una volta passato al secondo ordine ho dei processi in cui posso scambiare nergia con il campo magnetico assorbendo o emettendo 2 quanti di energia (o 2 $\hbar \omega$ ), per questo prendono il nome di processi a due fotoni.\\
Questi argomenti agli ordini superiori sono argomenti di ottica non lineare (il numero di fotoni coinvolti è maggiore di 1).

\subsection{Coefficienti di Einstein.}%
\label{sub:Coefficienti di Einstein.}
Vediamo adesso come può essere scritto l'elemento di matrice, per far questo possiamo metterci ad esempio nella Gauge di Coulomb, sostituiamo all'impulso:
\[
    \v{P} \to 
\v{P}- \frac{e}{n}\v{A}\cdot \v{P}
.\] 
In tale Gauge abbiamo che il potenziale diventa:
\[
    \hat{V}(t) = -\frac{e}{mc}\v{A}\cdot \v{P} + \ldots
.\] 
Dove non consideriamo il termine al secondo ordine. Il caso più interessante o semplice che possiamo trattare è quello in cui $\v{A}$ è una onda piana:
\[
    \v{A}=A_0 \hat{e}\cos(\v{k}\cdot \v{r}-\omega t) 
.\] 
Nel vuoto inoltre il campo elettrico può essere scritto come:
\[
\v{E}= -\frac{1}{c}\frac{\text{d} \v{A}}{\text{d} t} 
.\] 
Quindi possiamo riscrivere il $\hat{V}(t) $  come 
\[
    \hat{V}(t) = \frac{eE_0}{m\omega} \frac{1}{2}\left(e^{i(\v{k}\cdot \v{r}-\omega t)} + e^{-i(\v{k}\cdot \v{r}-\omega t) }\right) \hat{e}\cdot \v{P}
.\] 
Dove il termine $E_0$  è legato al potenziale vettore tramite:
\[
E_0=-A_0 \frac{\omega}{c}
.\] 
Quindi possiamo scrivere la probabilità con l'elemento di matrice esplicitato:
\[
    \overline{P}^{(1)}_{m\to n}=
    \frac{2\pi}{\hbar }\frac{e^2}{m^2\omega^2}E_0^2 \frac{1}{4}
    \left|
    \bra{\overline{n}}
    e^{\pm i \v{k}\cdot \v{r}}\hat{e}\cdot \v{P} 
    \ket{\overline{m}}
    \right|^2\rho(E_n = E_m \pm \hbar \omega) 
.\] 
Ci mettiamo adesso in una situazione in cui la lunghezza d'onda della radiazione è maggiore della estensione dei livelli che consideriamo (o anche del parametro della funzione d'onda che descrive i nostri livelli) :
\[
\lambda  \gg a
.\] 
ovvero $ka \ll 1$, questa è l'approssimazione di dipolo. \\
Tale approssimazione funziona bene anche per i solidi nelle ipotesi in cui $a$ è il passo reticolare. In questo modo si arriva alla espressione più classica per la regola d'oro di fermi per le transizioni nel campo elettromagnetico:
\[
    P^{(1) }_{m\to n}= 
    \frac{2\pi}{\hbar }\frac{e^2}{m^2\omega^2}\frac{E_0^2}{4}
    \left|\bra{n}\hat{e}\cdot \overline{P} \ket{m}\right|^2
    \rho(E_n=E_m \pm \hbar \omega) 
.\] 
Tipicamente l'elemento di matrice dato dal dipolo lo possiamo riscrivere sfruttando il commutatore tra $\v{r}$ ed $H_0$, in questo modo l'elemento di matrice diventa:
\[
    P^{(1) }_{m\to n}= 
    \frac{2\pi}{\hbar }\frac{e^2}{m^2\omega^2}\frac{E_0^2}{4}
    \left|\bra{n}\hat{r}\cdot \v{d} \ket{m}\right|^2
    \rho(E_n=E_m \pm \hbar \omega) 
.\] 
Possiamo adesso trattare un campo elettromagnetico non monocromatico, consideriamo la densità di energia:
\[
    u(\nu) d\nu = \frac{\epsilon E_0^2}{2} = n_\text{fot}\hbar \omega 
.\] 
Con $n_\text{fot} $ che è il numero di fotoni per unità di volume.\\
Possiamo adesso riscrivere la densità di stati congiunta come:
\[
    \rho(E_{mn}) = \frac{1}{2\pi\hbar }g(\nu_{nm}) 
.\] 
Mentre la probabilità non integrata di andare dallo stato m a quello n diventa:
\[
    P^{(1)}_{m\to n} =
    \frac{2\pi}{\hbar ^2}\left|\bra{n}\hat{n}\cdot \v{d} \ket{m}\right|^2
    \delta (\nu_{nm}\pm \nu) u(\nu) d\nu g(\nu_{nm}) 
.\] 
Nel caso in cui interagiamo con radiazione non polarizzata dobbiamo mediare il termine nel Bracket sulle orientazioni del dipolo (visto che viene fuori un $\cos^2$ alla fine della media resterà un fattore $1 /3$), quindi:
\[
    P^{(1)}_{m\to n} =
    \frac{2\pi}{3\hbar ^2}\left|\bra{n}d \ket{m}\right|^2
    \delta (\nu_{nm}\pm \nu) u(\nu) d\nu g(\nu_{nm}) 
.\]
Possiamo adesso integrare in $\nu$:
\[
    P^{(1) }_{m\to n}= \frac{2\pi}{3\hbar ^2}\left|\bra{n}d \ket{m}\right|^2
    u(\nu_{nm}) g(\nu_{nm}) 
.\] 
Se integriamo adesso su $\nu_{nm}$ considerando una $u$ molto larga rispetto alla $g$  (campo elettromagnetico estremamente non monocromatico) possiamo considerare la $g$ costante:
\[
    \overline{P}^{(1)}_{m\to n} =
    u(\nu_{nm}) B_{m\to n}
.\]
In cui il coefficiente $B$ vale:
\[
B_{m\to n} =
\frac{2\pi}{3\hbar ^2}\left|\bra{n}d \ket{m}\right|^2
.\] 
Che è un coefficiente introdotto da Einstein, introdotto assumendo che la probabilità di assorbimento e di emissione stimolata sia proporzionale alla densità di energia del campo elettromagnetico presente. \\
Possiamo notare che questa teoria è valida nel limite perturbativo con la regola d'oro di fermi e con un campo elettromagnetico molto largo rispetto alla distribuzione di energia dei livelli energetici (valida per campi elettromagnetici a banda larga).\\
Ovviamente i coefficienti $B_{m\to n}$ sono simmetrici, Einstein trovò questa simmetria con un sistema a due livelli all'equilibrio termodinamico con la radiazione, egli impose che tale sistema soddisfacesse la condizione di Boltzmann. In questo modo introdusse anche un coefficiente $A$ di emissione spontanea:
\[
P_{n\to m} = B_{n\to m} u + A
.\] 
Tale nuovo termine è indipendente dal campo elettromagnetico.\\
Dalla forma della equazione per la probabilità possiamo notare una simmetria tra la densità del campo elettromagnetico $u$ e la $g$ che è la densità di stati congiunta del sistema materiale, questo ci dice che la regola d'oro di fermi vale anche se abbiamo due stati ben definiti (su un singolo atomo) purché $\tau$ sia molto maggiore della larghezza di riga del campo elettromagnetico.\\
Possono anche esserci dei casi in cui la trattazione perturbativa non funziona a causa della ampiezza del campo magnetico, quindi non sarà più vero che
\[
    W^{(1)}\ll 1
.\] 
Vediamo in un sistema semplice cosa succede se non possiamo trattare il campo magnetico perturbativamente:
\begin{figure}[ht]
    \centering
    \incfig{livelliarimondo}
    \caption{Livelli energetici del sistema. In questo caso il livello 2 è il fondamentale.}
    \label{fig:livelliarimondo}
\end{figure}
Incidiamo con una onda elettromagnetica di ampiezza $E_0$ e frequenza $\omega$ mentre i due livelli sono separati da una energia $\hbar \omega_{12}$. La perturbazione $\hat{V}(t) $ sarà:
\[
    \hat{V}(t) = \frac{e}{m\omega} \frac{E_0}{2}\left(\hat{e}\cdot \v{p}\right)
    \left(e^{-i\omega t}+e^{i\omega t}\right)
.\] 
Gli elementi di matrice:
\[
\bra{1}W^+ \ket{2}= \frac{e}{m\omega}\frac{E_0}{2}\bra{1}\hat{e}\cdot \v{p}\ket{2}=
i \frac{E_0}{2}\hat{e}\cdot \v{d}_{12} 
.\] 
Nella quale $\v{d}_{12} = \bra{1}\v{d} \ket{2}$. L'altro elemento invece:
\[
\bra{2}W^+ \ket{1}=
i \frac{E_0}{2}\hat{e}\cdot \v{d}_{21} 
.\] 
Assumiamo inoltre la simmetria: $\v{d}_{12}=\v{d}_{21}$. Senza fare calcolo perturbativo cerchiamo di risolvere l'equazione di Schrodinger a due livelli sostituendovi la seguente $\psi$.
\[
    \ket{\psi(t) }=
    a_1(t) e^{-iE_1t /\hbar }\ket{1}+a_2(t) e^{-i E_2t /\hbar }\ket{2}
.\] 
Useremo inoltre l'approssimazione di onda rotante.
