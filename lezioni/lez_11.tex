\lez{11}{13-03-2020}{}
Passiamo alla Bose Einstein, il fatto di avere $\overline{n}_{q}$ medio molto maggiore di 1 mi da delle prop interessanti. \\
Per aumentare il n di partielle a T costante devo aumentare il pot chimico (shifto la curva in avanti), quindi $\mu $ deve tendere a 0. \\
Se $\mu \to 0$ quando $N\to \infty$ non ci sarebbe problema. Calcoliamo $N$:
\[
	N = \frac{4\pi V g \sqrt{2} m ^{3 /2}}{h^3}\int_{0}^{\infty} \frac{\mathcal{E} ^{1 /2}d\mathcal{E} }{\exp\left( \frac{\mathcal{E} -\mu }{kT} \right) -1} 
.\] 
Vediamo questo come diventa con $\mu = 0$, dovrebbe divergere $N$ per avere il risultato sperato, invece \footnote{Cambio variabile nell'integrale: $x = \frac{\mathcal{E} }{kT}$}:
\[
	N_{\text{max}}= \frac{4\pi V g \sqrt{2} m ^{3 /2}}{h^3} \left( kT \right)^{3 /2} \int_{0}^{\infty} \frac{x^{1 /2}dx}{e^{x}-1} 
.\] 
Sfortunatamente l'integrale non diverge, esso fa: $ \sqrt{\pi} /2 \cdot 2.612$. Possiamo allora devinire una densità critica
\[
	\left( \frac{N}{V} \right)_{\text{c}}= \frac{2.612}{\Lambda ^3}
.\] 
Se invece teniamo la densità $N /V$ costante ed abbassando la temperatura scopriamo che c'è anche una temperatura critica:
\[
	T_{c} = \frac{1}{2.31 m k} \left( \frac{N}{V} \right)^{2 /3} \frac{\left( 2\pi \hbar  \right) ^2}{\left( 4\pi \sqrt{2}  \right) ^{2 /3}}
.\] 
Dobbiamo capire perchè ci sono queste discrepanze dalla nostra attesa, dove finiscono le particelle che mancano? Vediamo prima di correggere la condensazione in un gas classico.\\
Per un gas classico si ha:
\[
	N = \frac{PV}{kT}
.\] 
Finchè vale si ha una relazione lineare tra pressione e numero di particelle, ma se continuiamo ad aumentare la densità ad un certo punto il gas condensa. La pressione a quel punto non cambia più (pressione di vapore). Per il gas di bosoni il ruolo della tensione è giocato dall'integrale nella \ldots:
\[
	y =\int_{0}^{\infty} \frac{\mathcal{E} ^{\frac{1}{2}d\mathcal{E} }}{\exp\left( \frac{\mathcal{E} -\mu }{kT} \right)-1 } 
.\] 
Ha esattamente lo stesso andamento della pressione del gas classico. Questo ci induce a pensare che anche il gas di bosoni stia transendo di fase. Quindi le particelle mancanti nel conto devono aver formato una nuova fase, tuttavia stiamo trattando un gas perfetto (particelle non interagenti). Come è possibile transire di fase? \\
Ma le particelle le abbiamo calcolate nel modo giusto, massando da somma a integrale:
\[
	N = \sum_{q}^{} \overline{n}_{q} \to \int\ldots
.\] 
Il criterio per questo passaggio era:  \[
	kT \gg \frac{\hbar^2}{2m}\left( \frac{2\pi}{L} \right) ^2
.\] Dove il secondo pezzo sono le separazioni tra i livelli energetici.
Per $T = T_{c}$ si ha che:
\[
	N^{2 /3} \gg \frac{2.31 \left( 4\pi\sqrt{2}  \right)^{2 /3}}{2}
.\] 
Che è sicuramente rispettata. Allora forse si tratta della densità di stati $\rho ( \mathcal{E} ) $, sappiamo che 
\[
	\rho ( \mathcal{E} ) \propto \mathcal{E} ^{1 /2}
.\] 
IN oltre sappiamo che $\rho \to 0$ con $\mathcal{E} \to 0$, tuttavia noi sappiamo che in questo caso vi è lo stato fondamentale. In questo caso nel limite $\mu \to 0$ si ha che diverge $\overline{n}_{q}( \Epsilon = 0 ) \to \infty$. Quindi la densità di stati va a zero ma esplode il numero di particelle. Per questo converge l'integrale! Quindi ci stiamo perdendo, con il conto tutte le particelle che stanno nel fontamentale $N^{*}$, allora  possiamo correggere 
\[
	N = N_0 + N^{*}
.\] 
con :
\[
	N^{*} = \int_{0}^{\infty} \frac{f( \mathcal{E} ) d\mathcal{E} }{\exp\left( \frac{\mathcal{E} }{kT} \right) -1} 
.\] 
Il numero medio di partielle nel fondamentale sarebbe:
\[
	\overline{n}_{0} = \frac{\rho }{\exp\left( -\frac{\ldots}{kT} \right) }
.\] sviluppando e \ldots\\
Questo fenomeno di chiama condensazione di Bose-Einstein.\\
Il numero $N$ che abbiamo trovato sopra corrisponde al numero di particelle negli stati eccitati:
\[
	N^{*} =.. 
.\] 
Con $T< T_{c}$ \ldots Possiamo allora riscrivere che :
\[
	N^{*} = N \left( \frac{T}{T_{c}} \right)^{3 /2}
.\] 
Sempre in queste temperature il numero di particelle nel fondamentale sarà:
\[
	N_0 = N\left( 1 -\left( \frac{T}{T_{c}} \right)^{3 /2}
 \right) 
.\] 
Sopra $T>T_{c}$ invece $N_0 \sim 0$ (nonostante sia sempre il più popolato esso è solo uno a fronte dei miliardi di stati eccitati!).\\
Per lo stesso motivo per cui $N = N_0 + N^{*}$, non è corretto dire che $\mu $ sia esattamente zero sotto a $T_{c}$, è circa zero.\\
Calcoliamo adesso l'energia:
\[
	E = \int_{0}^{\infty} \frac{\mathcal{E} \rho ( \mathcal{E} ) d\mathcal{E} }{\exp\left( \frac{\mathcal{E} }{kt} \right) -1} 
.\] 
Questa è corretta perchè le particelle nel fondamentale sono ad energia nulla! 
\[
	E = \frac{4\pi V g \sqrt{2} }{h^3}\left( m k T \right) ^{3 /2} kT \int_{0}^{\infty} \frac{x^{3 /2}}{e^{x}-1}dx  
.\] 
anche questo ultimo integrale converge, abbiamo quindi una forma per l'energia:
\[
	E = 0.770 NkT \left( \frac{T}{T_{c}} \right) ^{3 /2} \alpha T
.\] 
Quindi il calore specifico:
\[
	C_{V}= \left.\frac{\partial E}{\partial T} \right|_{V} = 1.9 nk \left( \frac{T}{T_{c}} \right) ^{3 /2}
.\] 
quindi il calore specifico segue l'andamento di $N^{*}$. 
Per $T > T_{c}$ Tenendo il numero di particelle costanti adesso non si può più mettere $\mu = 0$, vediamo che succede per $T\to \infty$, dobbiamotornare da Boltzmann.Ne risulta il seguente grafico tipico delle transizioni di fase.\\
Tornando al condensato possiamo calcolarci la pressione: 
 \[
	 P = \frac{2E}{3V} = 1.2 \frac{kT}{\Lambda ^3} \propto \left( kT \right) ^{5 /2} 
.\] 
Quindi non dipende più dal volume. (P costante per compressione isoterma, congruo conil fatto che abbiamo un cambio di fase)
\subsection{Oscillatore armonico}%
per quello classico:
\[
	H( p,q) = \frac{1}{2}m\omega ^2q^2 + \frac{1}{2m}p^2
.\] 
quindi se chiamiamo $\beta  = 1 /kT$:
\[
	Z( \beta )  = \frac{1}{h}\int_{-\infty}^{\infty} \int_{-\infty}^{\infty} \exp\left( -\beta \left( H( q,p)  \right)  \right) dqdp  
.\] 
Gli integrali in $p$ e $q$ sono separabili:
\[
	Z = \ldots = \frac{1}{\beta \hbar\omega } = 
.\] 
Se abbiamo $N$ oscillatori indipendenti allora si ha $Z_{t} = Z^{N}$ \footnote{Oscillatori indistinquigili, tipicamente va bene per i sistemi ideali.}.
\label{sub:oscillatore_armonico}


