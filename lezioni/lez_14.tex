\lez{14}{20-03-2020}{}
\subsection{Fluttuazioni nelle statistiche quantistiche}
\label{subsec:Fluttuazioni nelle statistiche quantistiche}
Abbiamo visto nella lezione 6 che le fluttuazioni del numero di particelle in un sistema può essere espresso come nella \ref{eq:flut-num-particelle}:
\[
	\overline{\left( \Delta N \right)^2} 
	=
	-kT
	\left.\frac{\partial ^2 \Omega }{\partial \mu ^2} \right|_{T,V}
.\] 
Il numero medio di particelle si esprime invece a partire dalla definizione di potenziale di Landau:
\[
	\overline{N}
	=
	- \left.\frac{\partial \Omega }{\partial \mu } \right|_{T,V}
.\] 
Se applichiamo queste al singolo stato quantistico $q$ avente la propria $\omega_q$ (energia) troviamo le fluttuazioni del numero di occupazione $\overline{n}$:
\[
	\overline{\left( \Delta n \right)^2}
	=
	kT \frac{\partial \overline{n}}{\partial \mu } 
.\] 
Sostituiamo a $\overline{n}$ le funzioni di distribuzione di BE (segno di sotto) oppure di FD (segno di sopra): 
\[
	\overline{n} 
	= 
	\frac{1}{e^{\left( \mathcal{E} - \mu  \right) /kT} \pm 1}
.\] 
E facciamo la derivata per $\overline{\left( \Delta n \right)^2}$:
\[\begin{aligned}
	\overline{\left( \Delta n \right) ^2}
	=&
	\frac{e^{\left( \mathcal{E} -\mu  \right) /kT}}
	{\left( e^{\left( \mathcal{E} -\mu  \right) /kT} \pm 1 \right)^2}=\\
	=&
	\frac{e^{\left( \mathcal{E} -\mu  \right) /kT} +1 - 1}
	{\left( e^{\left( \mathcal{E} -\mu  \right) /kT} \pm 1 \right)^2}=\\
	&=  
	\frac{1}{e^{\left( \mathcal{E} -\mu  \right) /kT} \pm 1 } \mp 
	\frac{1}{\left( e^{\left( \mathcal{E} -\mu  \right) /kT} \pm 1  \right)^2}=\\
	=& 
	\overline{n}_q \mp \overline{n}_q^2
.\end{aligned}\]
Per essere più espliciti possiamo scrivere che:
\[
	\overline{\left( \Delta n \right)^2} 
	=
	\begin{cases}
		&\overline{n}_q - \overline{n}_q^2 \quad \text{Fermi-Dirac} \\
		&\overline{n}_q + \overline{n}_q^2 \quad \text{Bose-einstein}
	\end{cases}
.\] 
Se la distribuzione fosse stata quella di Boltzmann avremmo ritrovato semplicemente:
\[
	\overline{\left( \Delta n\right)^2} = \overline{n}
\] 
Notiamo che le fluttuazioni sono maggiori nel caso della Bose-Einstein rispetto al caso della Fermi-Dirac. 
Nel caso della Fermi-Dirac si ha che $\overline{n}_q \le 1$ per il principio di esclusione di Pauli, quindi sicuramente $\overline{\left( \Delta n \right)^2} \ge 0$. \\
Per i bosoni invece si ha che le fluttuazioni sono maggiori che nel caso classico. 
Inoltre se $\overline{n}_q \gg 1$ allora per la Bose-Einstein $\overline{\left( \Delta n \right)^2} \propto \overline{n}_q^2$.
\subsection{Fluttuazioni del numero medio di particelle per un insieme di stati}
\label{subsec:Fluttuazioni del numero medio di particelle per un insieme di stati}
Prendiamo un insieme di $g$ stati aventi tutti la stessa energia (degenerazione nel numero di stati), le fluttuazioni del numero di particelle per questi sarà:
\[
	\overline{\left(\Delta N\right)^2} 
	=
	-kT \left.\frac{\partial ^2 \Omega_g}{\partial \mu^2} \right|_{T,V}
.\] 
Con $\Omega_g$ quello del gruppo di stati:
\[
	\Omega _g = \sum_{j=1}^{j=g} \Omega_j = g \Omega _q 
.\] 
Quindi possiamo scrivere:
\[\begin{aligned}
	\overline{\left( \Delta N \right) ^2} 
	=&
	g \overline{\left( \Delta n \right)^2}=\\
	=&
	g \left( \overline{n} \mp \overline{n}^2 \right) =\\
	=&
	\overline{N} \mp \frac{\overline{N}^2}{g} \label{eq:fluttuazioni-g}
.\end{aligned}\]
Notiamo quindi che l'importanza della correzione quantistica diminuisce man a mano che prendiamo un numero di stati isoenergetici sempre maggiore, tale correzione è massima quando consideriamo la singola celletta dello spazio delle fasi.
\subsection{Significato fisico della correzione quantistica alle fluttuazioni}
\label{subsec:Significato fisico della correzione quantistica alle fluttuazioni}
Prendiamo un sistema composto da un certo numero di fotoni, sappiamo che per questi non vale mai la conservazione del numero di particelle nel sistema, tuttavia studiando la seconda quantizzazione otteniamo che:
\[
	n = \left< \left| E \right|^2 \right>
.\] 
Con $n$ numero medio di fotoni e $E$ ampiezza del campo elettromagnetico. \\
Cerchiamo di quantificare questa ampiezza del campo elettromagnetico e trovarne le fluttuazioni quantistiche.\\
Supponiamo di avere il seguente apparato sperimentale:
\begin{figure}[H]
    \centering
    \incfig{sistema-di-emettitori-di-fotoni-con-rilevatore}
    \caption{Sistema di $N$ emettitori di fotoni con rilevatore, ogni emettitore ha una ampiezza di radiazione $\mathcal{E} $.}
    \label{fig:sistema-di-emettitori-di-fotoni-con-rilevatore}
\end{figure}
\noindent
Al rilevatore arriverà la radiazione di ogni sorgente, ognuna avente la propria fase. Ne risulterà un campo eletrico:
\[
	E = \sum_{j}^{N} \mathcal{E} e^{i \varphi_j}
.\] 
Quindi si ha anche che:
\[\begin{aligned}
	\left| E \right|^2 
	=&
	\sum_{j}^{N} \mathcal{E} e^{i \varphi_j} 
	\sum_{l}^{N} \mathcal{E} e^{-i \varphi_l}=\\
	=&
	N\mathcal{E} ^2 
	+
	\mathcal{E} ^2 \sum_{j \neq l}^{N} e^{i\left( \varphi_j - \varphi_l \right) }
.\end{aligned}\]
Se facciamo una media sull'Ensamble (muovere gli emettitori durante la misura per variare la fase) il secondo termine si cancella per interferenza (essendo le fasi Random), quindi ci resta:
\[
	\left<\left| E \right|^2 \right> 
	= 
	N \mathcal{E} ^2
.\] 
Per quanto assunto ad inizio sezione avremo anche che:
\[
	\left<n \right> 
	=
	N \mathcal{E}^2
.\] 
Sappiamo che la fluttuazione del numero di particelle può essere scritta come:
\[
	\frac{\left<\left( \Delta n \right) ^2 \right>}{\left< n \right>^2}
	= 
	\frac{\left<n^2 \right>-\left<n \right>^2}{\left<n \right>^2}
.\] 
Visto che $\left<n^2\right> = \left< \left| E \right| ^4 \right>$ ci conviene calcolarci questa quantità per trovare le fluttuazioni (riprendendo il modulo quadro che conteneva le fasi):
\[\begin{aligned}
	\left<\left| E \right| ^4 \right> 
	=&
	\mathcal{E}^4 
	\left<
	\left[ 
	N + 2 \sum_{j >  l}^{N} \cos( \varphi_j-\varphi_l) 
	\right]^2
	\right> =\\
	=&
	\mathcal{E}^4 
	\left[ 
	N^2 
	+
	\cancel{
	4N
	\left<	
	\sum_{j>l}^{N} \cos( \varphi_j - \varphi_l)  
	\right>
	}
	+
	4 
	\left<
	\sum_{j>l}^{N} \cos^2( \varphi_j-\varphi_l) 
	\right>
	\right]=\\
	=&
	\mathcal{E} ^4 
	\left[ 
	N^2 
	+
	4 \frac{N\left( N-1 \right)}{2} \frac{1}{2} 
	\right] =\\
	=&
	2N^2 \mathcal{E}^4 
.\end{aligned}\]
Nell'ultimo passaggio ci siamo messi nella situazione di avere tante sorgenti ($N\gg 1$).
In questo modo abbiamo che:
\[
	\frac{\left<\left( \Delta n \right) ^2 \right>}{\left< n \right>^2}
	=
	\frac{2N^2\mathcal{E}^4 - N^2\mathcal{E}^4}{N^2\mathcal{E}^4}
	=
	1
.\] 
In cocnlusione le fluttuazioni statistiche del numero medio di fotoni rilevate dal nostro rivelatore sono:
\[
	\left<\left( \Delta n \right)^2 \right> = \left<n \right>^2
.\] 
Confrontando questo termine con quanto osservato per la Bose-Eistein possiamo avere l'intuizione che, nel calcolo delle fluttuazioni, il termine $\left<n \right>^2$ sia associato alla natura ondulatoria della materia, infatti lo abbiamo ricavato in questa sezione ragionando sulla interferenza dei fotoni, il termine $\overline{n}$ invece può essere associato alla natura corpuscolare della materia.\\
Ipotizziamo di avere un sistema di particelle con $\overline{n}\gg 1$, in questo sistema il termine classico (o corpuscolare) di fluttuazioni può essere trascurato, se riuscissi a misurare questo numero di particelle otterrei un numero di particelle nel range:
\[\begin{aligned}
	n_\text{min} &= \overline{n}-\sqrt{\overline{\left( \Delta n \right)^2}} = 0\\
	n_\text{max} &= \overline{n}+\sqrt{\overline{\left( \Delta n \right)^2}} 
	= 2\overline{n}
.\end{aligned}\]
L'ampiezza di queste fluttuazioni può essere identificata da un parametro detto contrasto:
\[
	C = \frac{I_\text{max} - I_\text{min}}{I_\text{max} + I_\text{min}} =
	\frac{n_\text{max} - n_\text{min} }{n_\text{max} + n_\text{min} }
.\] 
Come si nota il contrasto può essere definito in termini di intensità (d'onda) oppure in termini di numero di particelle. Nel caso appena analizzato abbiamo ovviamente che $C=1$.\\
Se invece di considerare una sola celletta dello spazio delle fasi (adesso lo abbiamo implicitamente fatto prendendo fotoni tutti uguali) raccogliamo fotoni provenienti da $g$ celle diverse abbiamo visto che le fluttuazioni scalano con  $g$ dalla \ref{eq:fluttuazioni-g}, quindi avremo anche che:
\[\begin{aligned}
	&N_\text{min} = \overline{N}-\frac{\overline{N}}{\sqrt{g}}\\
	&N_\text{max} = \overline{N} + \frac{\overline{N}}{\sqrt{g}}
.\end{aligned}\]
Di conseguenza il contrasto sarà:
\[
	C = \frac{1}{\sqrt{g}}
.\] 
Quindi per riuscire a vedere le fluttuazioni quantistiche è necessario prendere fotoni che stiano il più possibile nella stessa cella nello spazio delle fasi, più coinvolgiamo fotoni da altre celle e meno saremo in grado di vedere tale effetto.
\begin{fact}[Controintuitività delle fluttuazioni quantistiche]{fact:Controintuitività delle fluttuazioni quantistiche}
	Più celle dello spazio delle fasi coinvolgiamo nel nostro esperimento e meno fluttuazioni saremo in grado di vedere. 
\end{fact}
\subsection{Numero di celle coinvolte nelle fluttuazioni del numero di fotoni}
\label{subsec:Numero di celle coinvolte nelle fluttuazioni del numero di fotoni}
Calcoliamo $g$ in funzione di parametri fisici del sistema per poter prevedere se le fluttuazioni quantistiche saranno significative nel numero dei fotoni nelle varie situazioni. Possiamo esprimere $g$ come volumetto dello spazio delle fasi:
\[
	g = \frac{\Delta \Omega V p^2 dp}{h^3}
.\] 
Con $\Delta \Omega $ frazione di angolo solido dalla quale noi raccogliamo fotoni, $V$ volume spaziale in cui si raccolgono fotoni e $p$ impulso di questi ultimi. \\
Fissiamo le idee con il seguente apparato preso in considerazione è il seguente:
\begin{figure}[H]
    \centering
    \incfig{fotoni-da-una-superficie-s-raccolti-a-s}
    \caption{Fotoni da una superficie S raccolti a S'}
    \label{fig:fotoni-da-una-superficie-s-raccolti-a-s}
\end{figure}
\noindent
Il "volume" di fotoni raccolti sarà dato da:
\[
	V = S cT
.\] 
Dove $T$ è il tempo in cui abbiamo rilevato i fotoni arrivare in $S'$. Sappiamo che per i fotoni vale:
\[
	p = \frac{h\nu }{c}
.\] 
Sappiamo inoltre che $\Delta p$ è legato quantisticamente alla larghezza spettrale della nostra sorgente  $\Delta \nu $, sfruttiamo la relazione tra questi due per riscrivere $g$.\\
I fotoni saranno suddivisi in pacchetti aventi un tempo di coerenza $\tau _c$, questo corrisponde al reciproco dell'ampiezza un frequenza secondo la trasformata di Fourier:
\[
	\Delta \nu \sim  \frac{1}{\tau_c}
.\] 
Con le ultime due possiamo riscrivere il differenziale dell'impulso come:
\[
	\Delta p = \frac{h}{c}\Delta \nu \sim \frac{1}{\tau _c}
.\] 
Riscriviamo il numero di cellette dello spazio delle fasi (o stati energetici dei fotoni coinvolti) $g$:  
\[\begin{aligned}
	g 
	=&
	\frac{\Delta \Omega V p^2 dp}{h^3}=\\
	=&
	\frac{\Delta \Omega ScT }{h^3} \frac{h^2}{\lambda ^2} \frac{h}{c \tau_c}  =\\
	=&
	\frac{\Delta \Omega S}{\lambda^2}\frac{T}{\tau_c}
.\end{aligned}\]
In questa possiamo notare il rapporto tra il tempo di coerenza dei fotoni ed il tempo di osservazione, per avere $g$ più piccolo possibile sarà intanto necessario rendere questo rapporto il più vicino possibile all'unità.\\
L'ordine di grandezza dei tempi di corenza è:
\begin{itemize}
	\item Per una lampada $\tau _c \sim $ ns, quindi $\Delta \nu \sim 10^9$ Hz.
	\item Per un laser $\tau_c \sim $ ms e $\Delta \nu \sim 10^3$ Hz.
\end{itemize}
Concentriamoci adesso sulla parte spaziale dell'espressione e vediamo se può essere scritta in modo da darci informazioni utili su $g$.\\
Riscriviamo $\Delta \Omega $ in funzione dei parametri fisici del nostro apparato:
\[
	\Delta \Omega \propto \frac{S'}{R^2}
.\] 
Introduciamo anche un parametro superficiale detta \textit{Estensione del fascio}:
\[
	U = \Delta \Omega S
.\] 
Questa estensione, vista la $\Delta \Omega $ scritta sopra, può essere espresso come:
\[
	U = \frac{SS'}{R^2}
.\] 
La sorgente di superficie $S$ solitamente avrà una espansione angolare del tipo \footnote{Questo è l'angolo solido sotto al quale tipicamente viene sparpagliata la radiazione dalla sorgente.}:
\[
	\Delta \Omega \sim \frac{\lambda ^2}{S}
.\] 
Quindi abbiamo una estensione del fascio critica $U_c$:
\[
	U_c \sim \lambda ^2
.\] 
Un rivelatore che riesca a raccogliere fotoni con una estensione di queste dimensioni sicuramente raccoglierà fotoni aventi tutti la stessa energia poichè provenienti dallo stesso pacchetto coerente, vediamo infatti che la $g$ si riscrive come:
\[
	g=\frac{U}{U_c}\frac{T}{\tau _c}
.\] 
Abbiamo quindi un numero di celle che è frutto del prodotto di una coerenza spaziale ed una coerenza temporale.\\

\subsection{Esempio numerico di numero di celle occupate.}
\label{subsec:Esempio numerico di numero di celle occupate.}
Quindi abbiamo che la condizione $U / U_c \sim 1$ equivale alla condizione che si ha per vedere le frange di interferenza nell'esperiento di Young.\\
Facciamo un esempio numerico: 
se prendo $\lambda \sim $ nm ho che $U_c \sim 10^{/10}$ cm$^2$.
Se prendiamo una lampadina di dimensioni superficiali del  $\sim $ mm$^2$ ed un rivelatore di $\sim $cm$^2$ ed otteniamo $U\sim 10^{-6}$ cm$^2$. 
Di conseguenza in questo caso avremmo nella migliore delle ipotesi $U / U_c \sim 400 \sim g$. Possiamo scordarci le frange di interferenza con 400 cellette occupate.\\

\subsection{Coerenza dei fotoni ed esperimento di Young}
\label{subsec:Coerenza dei fotoni ed esperimento di Young}
Nell'esperimento di Young, schematizzato in Figura \ref{fig:esperimento-di-young} il rapporto $U /U_c$ si scrive come:
\[
	\frac{U}{U_c} = \frac{S'}{A_c}
.\] 
Dove $A_c$ è una superficie critica che ricaviamo sotto.
\begin{figure}[H]
    \centering
    \incfig{esperimento-di-young}
    \caption{Esperimento di Young, $P_1$ e $P_2$ sono le due fenditure.}
    \label{fig:esperimento-di-young}
\end{figure}
\noindent
La condizione per riuscire a vedere le frange di interferenza è:
\[
	\Delta \theta S \lesssim \lambda 
.\] 
Questo significa che i punti $P_i$ devono stare all'interno di un area critica per poter vedere le frange di interferenza, questa area critica è:
\[
	A_c \simeq \left( R \Delta \theta  \right)^2 =
	\frac{R^2 \lambda ^2}{S^2}
.\] 

\subsection{Ottenere la coerenza dalla radiazione stellare}
\label{subsec:Ottenere la coerenza dalla radiazione stellare}
Se si utilizza come sorgente la radiazione stellare si hanno $\Delta \Omega $ (quindi anche $U$) molto più piccoli rispetto a qualunque sorgente che possiamo costruire sulla terra.\\
Con queste sorgenti possiamo avere una estensione del fascio simile a quella di coerenza. L'idea di questi esperimenti era la seguente:
\begin{figure}[H]
    \centering
    \incfig{esperimento-per-raccogliere-la-radiazione-stellare}
    \caption{Esperimento per raccogliere la radiazione stellare: le due stelle sono in realtà una sola (vista da sue fotomoltiplicatori a distanda $d$).}
    \label{fig:esperimento-per-raccogliere-la-radiazione-stellare}
\end{figure}
\noindent
Per poter vedere interferenza dal nostro apparato dobbiamo avere:
\[\begin{aligned}
	 d 
	 <&
	 \sqrt{A_c} =\\
	 =& \frac{2}{\sqrt{\pi}}\frac{R\lambda }{a}
.\end{aligned}\]
In questo modo è possibile una misura del rapporto tra la distanza e le dimensioni della stella, anche detto parametro angolare della stella:
\[
	\alpha = \frac{a}{R}
.\] 
Per migliorare la misura è possibile costruire due fotomoltiplicatori che raccolgono la radiazione dalla stella e la amplificano verso il rilevatore nel percorso di Figura \ref{fig:esperimento-per-raccogliere-la-radiazione-stellare}.
\subsection{Correlazione tra fasci di particelle: esperimento di Hanbury Brown e Twiss.}
\label{subsec:Correlazione tra fasci di particelle}
Gli esperimenti sulla radiazione stellare vennero resi più sofisticati da Hanbury Brown e Twiss che studiarono la radiazione proveniente da Sirio.\\
Il loro scopo era una misura della correlazione tra i fotoni provenienti da questa stella raccolti da due fotomoltiplilcatori disposti sempre come in Figura \ref{fig:esperimento-per-raccogliere-la-radiazione-stellare}. \\
Se nominiamo i due fotomoltiplicatori con il pedice $_1$ e  $_2$ allora possiamo definire la correlazione tra il numero dei conteggi di fotoni di questi due strumenti come:
\[
	C = \overline{\left( N_1 - \overline{N}_1\right) \left( N_2-\overline{N}_2 \right)}
.\] 
Supponiamo di essere all'interno del diametro di coerenza avremo che $g = T /\tau_c$, quindi sappiamo che le fluttuazioni di misura dei fotomoltiplicatori saranno:
\[\begin{aligned}
	&\overline{\left( N_1-\overline{N}_1 \right)^2} 
	= 
	\overline{N}_1 + \frac{\left( \overline{N}_1 \right)^2}{g}\\
	&\overline{\left( N_2-\overline{N}_2 \right)^2} 
	= 
	\overline{N}_2 + \frac{\left( \overline{N}_2 \right)^2}{g}
.\end{aligned}\]
Anche il numero totale di fotoni avrà la stessa tipologia di fluttuazioni:
\[
	\overline{\left( N_1+N_2-\overline{N}_1-\overline{N}_2 \right)^2}
	=
	\overline{N}_1 + \overline{N}_2
	+
	\frac{\left( \overline{N}_1 + \overline{N}_2 \right)^2}{g}
.\] 
Sviluppiamo adesso i quadrati a destra e semplifichiamo quest'ultima esperessione:
\[\begin{aligned}
	\cancel{\overline{\left( N_1-\overline{N}_1 \right)^2}} 
	+
	\cancel{\overline{\left( N_2-\overline{N}_2 \right)^2}}
	+ 2 
	\overline{\left( N_1-\overline{N}_1 \right) \left( N_1-\overline{N}_2 \right) }
	=
	\cancel{\overline{N}_1} + \cancel{\overline{N}_2}
	+
	\cancel{\frac{\overline{N}_1^2}{g}} + \cancel{\frac{\overline{N}_2^2}{g}}
	+
	2 \frac{\overline{N}_1 \overline{N}_2}{g}
.\end{aligned}\]
Abbiamo in conclusione che la correlazione tra i numeri medi registrati dai due fotomoltiplicatori sarà:
\[
	C = \frac{\overline{N}_1 \overline{N}_2}{g} 
.\] 
Ricordiamo che in questo caso abbiamo assunto la totale coerenza spaziale, quindi la $g$ nella formula sarà la sola coerenza temporale. \\
Potremmo anche fare il caso opposto: teniamo la distanza tra i fotomoltiplicatori costanti ed inseriamo una asincronicità $\tau $ nella misura tra i due strumenti, in questo modo la correlazione sarebbe:
\[
	\overline{C}( \tau ) = 
	\overline{\left( N_1(\tau ) - \overline{N}_1 \right)
	\left( N_2( \tau + \overline{c}) - \overline{N}_2 \right) }
.\] 
Inoltre nel caso dei fermioni la correlazione media ha il segno opposto a quella calcolata adesso, questo significa che se un rilevatore riceve un numero superiore alla media di fermioni allora l'altro sicuramente ne riceve meno, viceversa con i bosoni.
\subsection{Dinamica delle flutuazioni: Rumore Jonshon nei circuiti elettrici}
\label{subsec:Dinamica delle flutuazioni: Rumore Jonshon nei circuiti elettrici}
Misurano una differenza di potenziale ai capi di una resistenza $R$ con un multimetro molto sensibile siamo in grado di rilevare delle piccole oscillazioni del valore misurato, queste corrispondono al rumore Jonshon.\\
Le misure effettuate da Jonshon concluserò che, per uno strumento che misura in una ampiezza di banda $\Delta f$ si ha:
\[
	\overline{V^2}_{\Delta f} 
	=
	4Rk_BT\Delta f
.\] 
Notiamo che, se non si applica esternamente alcuna ddp al sistema si ha anche che ($\overline{V}=0$):
\[
	\overline{V^2}_{\Delta f}
	= 
	\overline{\left( \Delta V \right) ^2}
	=
	\overline{V^2}
.\] 
Il fenomeno ha una dimostrazione macroscopica dovuta a Nyquist che aggira il conto statistico. Per far questo Nyquist considera la seguente linea di trasmissione:
\begin{figure}[H]
    \centering
    \incfig{linea-di-trasmissione-di-nyquist}
    \caption{Linea di trasmissione di Nyquist}
    \label{fig:linea-di-trasmissione-di-nyquist}
\end{figure}
\noindent
L'ipotesi di Nyquist consisteva nel considerare le resistenze all'equilibrio termodinamico con il campo elettromagnetico presente nella linea. In queste ipotesi la potenza del campo elettrico all'interno della linea la si ottiene mediante la teoria di corpo nero.\\
All'equilibrio tanta potenza viene assorbita dalle resistenze tramite effetto Joule quanta ne viene emessa dal campo elettromagnetico all'interno della linea. 
Quindi anche le resistenze saranno in grado di emettere radiazione per effetto Joule, questo effetto sarà allora caratteristico della agitazione termica del materiale che compone la resitenza stessa. \\
Imponendo quindi l'equilibrio sul sistema troveremo la potenza dissipata per effetto joule dal sistema e questa sarà proprio la potenza di rumore.\\
Consideriamo le onde EM all'interno della linea di trasmissione, la loro lunghezza d'onda sarà quantizzata:
\[
	L = n \frac{\lambda}{2}
.\] 
con $\lambda = c /\nu $, $c$ è la velocità di propagazione nella linea. La lunghezza della linea di trasmissione sarà data da:
\[
	L = n \frac{c}{2\nu }
	\implies
	\nu = n \frac{c}{2L}
.\] 
Quindi in un intervallo di frequenze $\Delta \nu $ avremo un numero di modi:
\[
	m = \frac{2L}{c}\Delta \nu 
.\] 
Ogni modo alla frequenza $\nu $ avrà una energia:
\[
	h\nu \overline{n} = h\nu \frac{1}{e^{h\nu /kT}-1}
.\] 
Il rumore Jonshon si manifesta a basse frequenze, possiamo tranquillamente assuere $h\nu /kT \ll 1$. Sviluppiamo allora la Bose-Einstein:
\[
	h\nu \overline{n} \sim kT
.\] 
L'energia contenuta nell'intervallo di frequenze $\Delta \nu $ sarà:
\[
	E_{\Delta \nu } \approx k_BT \frac{2L}{c}\Delta \nu 
.\] 
Il tempo che la radiazione impiega ad arrivare alle due resistenze sarà $t\approx L /c$, quindi abbiamo che la potenza media che arriva alle due resistenze è:
\[
	P_{\Delta \nu }= k_B T\Delta \nu 
.\] 
Questa deve essere la stessa che la nostra resistenza genera per effetto Joule:
\[
	I^2R = k_BT\Delta \nu 
.\] 
Quindi il potenziale $V_{\Delta \nu}$ di rumore nella nostra resistenza deve essere:
\[
	V_{\Delta \nu } = 2RI
.\] 
Sostituendo con l'uguaglianza sopra abbiamo:
\[
	\overline{V_{\Delta \nu }^2}= 4Rk_BT\Delta \nu 
.\] 
Che è proprio il risultato che si ottiene anche sperimentalmente.

