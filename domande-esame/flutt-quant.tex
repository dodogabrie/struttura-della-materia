\clearpage 
\section*{Domande di esame per il capitolo}%
\begin{quest}[Cosa cambia tra fluttuazioni classiche e quantistiche]{quest:Cosa cambia tra fluttuazioni classiche e quantistiche}
    Nelle fluttuazioni quantistiche (del numero di particelle) abbiamo un termine aggiuntivo dovuto alle statistiche differenti:
    \[
        \overline{\left(\Delta N\right)^2} = \overline{N}\pm \frac{\overline{N^2}}{g}
    .\] 
    Dove il $+$  vale per Bosoni, il $-$  per fermioni. $g$  è il numero di celle dello spaz. delle fasi che considero. È importante ricordare che questo risultato si ottiene passando da:
    \[
        \overline{\left(\Delta N\right)^2} = -kT \left.\frac{\partial ^2 \Omega}{\partial \mu^2} \right|_{T,V}
    .\] 
    E visto che 
    \[
        \overline{N} = -\left.\frac{\partial \Omega}{\partial \mu} \right|_{T,V}
    .\] 
    Si arriva a 
    \[
        \overline{\left(\Delta n\right)^2}= kT\frac{\partial \overline{n}}{\partial \mu} 
    .\] 
    Per il numero di occupazione, per un insieme di stati isoenergetici basterà inserire:
    \[
        \overline{\left(\Delta N\right)^2} = g \overline{\left(\Delta n\right)^2}
    .\] 
    Dal punto di vista fisico il termine $\pm$ deriva dalla voglia che queste particelle hanno di stare nello stesso stato: per fermioni sappiamo infatti che $\overline{n}< 1$, per Bosoni $\overline{n}\gg 1$. 
    Per un gas di Bosoni mi aspetto di vedere delle grosse fluttuazioni se sono a basse temperature, dove i numeri di occupazione sono alti.
\end{quest}

