\lez{6}{28-02-2020}{}
\subsection{Fluttuazioni}%
Parliamo adesso delle fluttuazioni, abbiamo detto che il nostro metodo funziona nel limite in cui i valori medi ed i più probabili delle nostre grandezze termodinamiche coincidono e che la distribuzione di probabilità sia estremamente piccata nel punto di maggiore probabilità.\\
Quindi cerchiamo di stimare quali sono il limiti di applicabilità del nostro modello. Indaghiamo inoltre qual'è la probabilità di trovare una fluttuazione: una differenza rispetto al valore medio atteso.\\
\paragraph{Fluttuazioni dell'energia.}%
Ad esempio possiamo isolare il nostro sistema, fin'ora all'equilibrio con il bagno termico, misuriamo la sua energia. Ci aspettiamo di trovare sempre la stessa energia tranne qualche fluttuazione che adesso andiamo appunto a stimare. \\
Abbiamo detto che l'energia totale è
\[
	E = \frac{1}{2m}\sum_{1}^{3N} p_{i}^2
.\] 
Ipotizziamo adesso di fissare una energia per il sistema. Come abbiamo visto, nello spazio delle fasi, fissare una energia significa fissare una superficie sferica \footnote{Una sfera 3N dimensionale}, il raggio di questa superficie sarà dato da $P^{*}$ tale che:
\[
	\left( P^* \right)^2 = 2mE = \sum_{i}^{} p_{i}^2
.\] 
La probabilità di trovare il nostro sistema con una certa energia è la stessa della scorsa lezione:
\[
	W\left( E \right) = \frac{1}{Z}\rho\left( E \right) \exp\left( -\frac{E}{kT} \right) 
.\] 
dove $\rho\left( E \right)$ è tale che il volume della buccia spessa $\delta E$ intorno alla energia $E$ nello spazio delle fasi che stiamo considerando, questa densità rispetta la relazione:
\[
	\rho\left( E \right) \delta E \propto \delta V^*
.\]
\begin{figure}[H]
    \centering
    \incfig{calcolo-delle-fluttuazioni}
    \caption{Calcolo delle fluttuazioni: buccia di spessore $\delta E$ nello spazio delle fasi.}
    \label{fig:calcolo-delle-fluttuazioni}
\end{figure}
\noindent
Quindi se il raggio di tale sfera nello spazio delle fasi è $P^*$ allora possiamo considerare il volume sotteso a tale raggio come  $V^{*}$:
\[
	V^* \propto \left( P^* \right)^{3N}
.\] 
Se consideriamo un $\delta E$ sufficientemente piccolo allora possiamo sviluppare $\delta V^{*}$:
\begin{align}
	\rho\left( E \right) \delta E \propto& \frac{\partial V^*}{\partial E}\delta E  \propto\\
	\propto &\left( P^* \right) ^{3N-1}\frac{\partial P^*}{\partial E} \delta E \propto\\
	\propto &\left( E^{1 / 2} \right) ^{3N-1}E^{- 1 / 2} \delta E
.\end{align}
Dove nell'ultimo passaggio si è sfruttato il fatto che $\left( P^{*} \right) \propto E$. Si ottiene la seguente relazione: 
\[
	\rho\left( E \right) \delta E \propto E ^{\frac{3N}{2} -1}\delta E 
.\] 
Questo risultato è coerente con il fatto che inserendo una singola particella ri-otteniamo $\rho\left( E \right) \propto E^{1 /2}$ come discusso nella lezione precedente.\\
Quindi il nostro $W \left( E \right) $ può essere scritto come:
\[
	W\left( E \right)  = C^{N} E^{\left( \frac{3N}{2}-1 \right)}\exp\left( -\frac{E}{kT} \right) 
.\] 
La costante $C^{N}$ dipende dal numero di particelle. La si ottiene tramite:
\[
	1 = \int W\left( E \right) dE
.\] Ciò che si otterrebbe da questo conto è molto simile a quanto fatto nel caso di singola particella:
\[
	\frac{1}{C^{N}}= \Gamma \left( \frac{3N}{2} \right) \left( kT \right) ^{\frac{3N}{2}}
.\] 
e mettendo tutto indieme:
\[
	W\left( E \right) = \frac{1}{\Gamma\left( \frac{3N}{2} \right) }\left( \frac{E}{kT} \right) ^{\left( \frac{3N}{2}-1 \right) }\cdot 
	\frac{1}{kT}\exp\left( -\frac{E}{kT} \right) 
.\] 
Il valore più probabile lo avremo quando la probabilità ha un massimo rispetto all'energia, quindi:
\[
	0 = \frac{\partial }{\partial E} \left[ W\left( E \right) \right] \propto \frac{\partial }{\partial E} \left[ E^{\frac{3N}{2}-1} \exp\left( -\frac{E}{kT} \right) \right] 
.\] 
Risolvendo il problema algebrico di minimo si conclude che il massimo di $W\left( E \right)$ lo si ha per il valore di energia $E_{\text{max}}$ tale che:
\[
	E_{\text{max}} = \left( \frac{3N}{2}-1 \right) kT
.\] 
Se il nostro modello è corretto questo sarà anche il valore per cui la nostra distribuzione $W\left( E \right)$ è particolarmente piccata. \\
Cerchiamo di caratterizzare tale distribuzione, il valore medio dell'energia si trova con:
\[
	\overline{E} = \int E W\left( E \right) dE = C_{N} \int E^{\frac{3}{2}N}\exp\left( -\frac{E}{kT} dE \right) 
.\] 
Cambiando variabili come nel caso di singola particella ($E /( kT )=x$), questa volta si ottiene:
\[
	\overline{E} = C_{N} \left( kT \right) ^{\frac{3}{2}N-1} \Gamma \left( \frac{3}{2}N +1 \right) = \frac{3}{2}N kT
.\] 
Quindi nel limite $N\gg 1$ energia media e massimo della distribuzione coincidono. Non si può dire lo stesso per $N= 1$.
\begin{figure}[H]
    %This is a custom LaTeX template!
    \centering
    \incfig{andamento-della-distribuzione-di-energia-al-variare-di-n}
    \caption{\scriptsize Andamento della distribuzione di energia al variare di N, notiamo che nella funzione scritta al centro nel caso di una singola particella si scrive con $\mathcal{E}$, non con $E$.}
    \label{fig:andamento-della-distribuzione-di-energia-al-variare-di-n}
\end{figure}
\noindent
Abbiamo già scoperto un limite della nostra trattazione, possiamo applicare il modello termodinamico di passaggio al continuo dell'energia solo nel caso di sistemi aventi $N \gg 1$. Altrimenti la distribuzione in energia non può essere considerata "piccata". Ci concentriamo allora adesso su sistemi aventi un grande numero di particelle per poter procedere con la nostra trattazione.\\
Cerchiamo di quantificare le fluttuazioni: possiamo vedere le fluttuazioni come la larghezza della curva piccata. Per far questo semplifichiamo la notazione sul valore medio: $\overline{E} = E$.
\[
	E =  \overline{E} = \int E W( E ) dE
.\] 
Quindi l'energia del sistema che misuriamo una volta staccato dal bagno termico la chiamiamo $\mathcal{E}$ \footnote{La notazione è cambiata rispetto alla scorsa lezione: adesso questa $\mathcal{E}$ è l'energia totale del sistema misurata, non l'energia della singola particella.}, ed è vero anche che $\overline{\mathcal{E}}= E$. La fluttuazione in questa notazione si scrive come:
\[
	\Delta E = \mathcal{E}- E 
.\] 
facendo il valore medio di $\Delta E $ otteniamo zero, quindi facciamone il valor medio quadro:
\[
	\overline{\left( \Delta E \right) ^2} = \overline{\left( \mathcal{E} - E \right)^2 } = \overline{\mathcal{E}^2- 2\mathcal{E} E + E^2}= 
	\overline{\mathcal{E}^2} - 2 E^2 + E^2 = \overline{\mathcal{E}^2} - E^2
.\] 
Risultato in linea con le basi di statistica. \\
Il valore $\overline{\mathcal{E}^2}$ è dato da:
\[
	\overline{\mathcal{E}^2} = \int \mathcal{E}^2 W( \mathcal{E}) d\mathcal{E}
.\] 
Potremmo metterci a fare il conto direttamente con l'energia, tuttavia possiamo utilizzare un metodo più facile che consiste nel considerare la funzione di partizione. Visto che la funzione di partizione è la somma di tutte le probabilità essa tiene di conto della forma della distribuzione di energia. Inoltre ha un legame molto utile con l'energia libera, procediamo quindi a calcolarla. Nella approssimazione continua tale funzione prende la forma:
\[
	Z = \int\rho\left( \mathcal{E} \right) \exp\left( -\frac{\mathcal{E}}{kT} \right) d\mathcal{E}
.\] 
Con l'energia libera che si scrive in funzione di Z:
\[
	F = -kT \ln Z
.\]
Facciamo la derivata seconda di questa funzione rispetto alla temperatura:
\begin{align}
	\frac{\partial ^2 F}{\partial T^2} =& \frac{\partial ^2}{\partial T^2} \left( -kT \ln Z \right) =\\
	=&\frac{\partial }{\partial T} \left( -k\ln Z -kT \frac{1}{Z} \frac{\partial Z}{\partial T}  \right) =\\
	=&-k \frac{1}{Z}\frac{\partial Z}{\partial T} 
		- k \frac{1}{Z}\frac{\partial Z}{\partial T} 
		+ kT \frac{1}{Z^2}\left( \frac{\partial Z}{\partial T}  \right) ^2 
		- \frac{kT}{Z}\frac{\partial ^2 Z}{\partial T^2} 
.\end{align}
Quindi inserendo la Z scritta sopra e facendone le derivate \footnote{Provando si nota che la derivata seconda della Z è nulla.} otteniamo:
\[
	\frac{\partial ^2 F}{\partial T^2} = \frac{1}{kT^3} 
	\left[
		\frac{\left( \int \mathcal{E}\rho(\mathcal{E})\exp\left( -\frac{\mathcal{E}}{kT}  \right)d\mathcal{E} \right)^2  }{Z^2}  
		- \frac{1}{Z}\int \mathcal{E}^2\rho\left( \mathcal{E} \right) \exp\left( - \frac{\mathcal{E}}{kT} \right)d\mathcal{E}  
	\right]
.\] 
Riconosciamo i due termini nell'integrale, formano proprio la differenza che stavamo cercando: le fluttuazioni \footnote{A meno di un segno.}.
\[
	\frac{\partial ^2 F}{\partial T^2} = \frac{E^2-\overline{\mathcal{E}^2}}{kT^3}
.\] 
Quindi ricordando come si definisce la \hyperref[eq:capacita-termica]{Capacità termica} a volume costante si ottiene anche che: 
i
\[
	\overline{\left( \Delta E \right) ^2} = -kT^3 \frac{\partial ^2 F}{\partial T^2} = kT^2 C_{v}
.\] 
Inoltre ricordando che abbiamo derivato sia l'energia media:
\[
	E = \frac{3}{2}N kT
.\] 
Che la capacità termica:
\[
	C_{v} = \left.\frac{\partial E}{\partial T} \right|_{V} =  \frac{3}{2}Nk
.\] 
Possiamo calcolarci la fluttuazione e normalizzarla sull'energia media:
\[
	\frac{\sqrt{\overline{\left( \Delta E \right) ^2}} }{E} = \frac{\sqrt{kT^3 \frac{3}{2}Nk} }{\frac{3}{2}NkT} = \frac{\sqrt{N} }{\sqrt{\frac{3}{2}}N}\sim \frac{1}{\sqrt{N} }
.\] 
L'ampiezza relativa delle fluttuazioni scala come $N^{-1 /2 }$. coerente con il fatto che la larghezza della distribuzione decresce con il numero di particelle.\\
\paragraph{Fluttuazioni del numero di particelle.}%
Potremmo assumere che il nostro sistema possa scambiare particelle con il bagno termico, in tal caso potremmo calcolare l'andamento delle fluttuazioni del numero di particelle. 
Per calcolare questa fluttuazione si procede in modo del tutto analogo a quanto fatto con l'energia ma si considera (anzichè l'energia libera) il \hyperref[eq:potenziale-Landau1]{Potenziale di Landau} \footnote{Naturalmenta a P e V costanti.} facendo la derivata rispetto al Potenziale chimico.
\[
	\left( \frac{\partial ^2 \Omega}{\partial \mu^2}  \right)  = \frac{\partial ^2}{\partial \mu^2} \left[ -kT \ln \left( \sum_{\alpha}^{} \exp\left( -\frac{E_{\alpha}-\mu N_{\alpha}}{kT} \right)  \right)  \right] 
.\] 
Quindi analogamente a sopra si ottiene:
\[
	\frac{\partial ^2\Omega}{\partial \mu^2} = -\frac{1}{kT}\left( \overline{N_{\alpha}^2}- N^2 \right) \implies
	\overline{\left( \Delta N \right) ^2} = - kT \left( \frac{\partial ^2\Omega}{\partial \mu^2}  \right) _{T,V} \label{eq:flut-num-particelle}
.\] 
visto che il numero medio di particelle è definito da:
\[
	N = -\left.\frac{\partial \Omega}{\partial\mu} \right|_{T,V}
.\] 
Riscriviamo la fluttuazione di particelle come:
\[
	\overline{\left( \Delta N \right) ^2} = kT \left.\frac{\partial N}{\partial \mu} \right|_{T,V}
.\] 
Ricordiamo che $\mu$ è una funzione di (P,T), e le sue derivate ci permettono di scrivere, introducento la densità di particelle come $\rho = N /V$: 
\[
	\left.\frac{\partial \mu}{\partial N} \right|_{T,V} = \frac{1}{N}\left( \frac{\partial P}{\partial \rho}  \right) _{T,V}
.\] 
Ma $\left.\frac{\partial P}{\partial \rho} \right|_{T,V}$ è legato alla $k_{T}$ :
\[
	k_{T} = \frac{1}{V}\left( \frac{\partial V}{\partial P}  \right) _{T} = \frac{1}{\rho}\left( \frac{\partial \rho}{\partial P}  \right) _{T}
.\] 
quindi si ha:
\[
	\overline{\left( \Delta N \right) ^2}= NkT \rho k_{T}
.\]
Con una fluttuazione relativa:
\[
	\frac{\sqrt{\overline{\left( \Delta N \right) ^2}} }{N} = \frac{kT}{V}k_{T}
.\] 
\paragraph{Esempio: gas ideale}%
Nel caso di gas ideale si ha $PV=NkT$, quindi la compressibilità si ottiene tramite:
\[
	k_{T} =- \frac{1}{V}\frac{\partial }{\partial P} \left( \frac{NkT}{P} \right) = \frac{NkT}{P^2V}=   \frac{1}{P} 
.\] 
Che sostituito nella fluttuazione relativa ci da lo stesso risultato che abbiamo ottenuto per la energia:
\[
	\frac{\sqrt{\overline{\left( \Delta N \right) ^2}} }{N} = \frac{1}{\sqrt{N} }
.\] 
Quindi la fluttuazione di energia relativa e la fluttuazione di particelle relativa vanno allo stesso modo nel caso di gas ideale.\\
Notiamo che concettualmente le due fluttuazioni sono diverse, per la fluttuazione di energia abbiamo trovato tale quantità fissando volume e numero di particelle, nel secondo caso abbiamo fissato soltanto il volume. Tuttavia nel secondo caso oltre alla variazione del numero di particelle ci sara anche una variazione in energia e quindi una fluttuazione dell'energia del sistema!\\
Non possiamo assumere che quest'ultima fluttuazione sia la stessa che nel caso in cui il numero di particelle sia fissato perchè altrimenti la variazione del numero di particelle non influirebbe con la variazione di energia e questo risulta impossibile in generale \footnote{Ad esempio quando interagiscono l'una con l'altra tramite un potenziale l'energia dipende dal numero di particelle interagenti.}.\\
Spesso troveremo fluttuazioni di quantità termodinamiche strettamente connesse tra loro, questo giocherà a nostro favore perchè potremo ricavare l'una dall'altra. Un esempio è la fluttuazione di temperatura.
\paragraph{Fluttuazioni della temperatura}%
Per un sistema con numero di particelle e volume costante si ha:
\[
	\Delta T = \left.\frac{\partial T}{\partial E} \right|_{V,N}\Delta E= \frac{\Delta E}{c_{V}}
.\] 
\[
	\overline{\left( \Delta T\right) ^2} = \frac{kT^2}{c_{V}}
.\] 
\subsection{Fluttuazioni dipendenti ed indipendenti}%
Possiamo anche trovare delle espressioni più generali per le fluttuazioni partendo dallo sviluppo gaussiano delle distribuzioni discusso nella \hyperref[eq:Gauss-approx]{Lezione 4}. \\
Supponiamo ci interessi trovare il nostro sistema in una determinata grandezza fisica x e di voler prevedere le fluttuazioni del sistema dal valore centrale di quella grandezza. Dobbiamo partire dalla dipendenza dell'entropia da quella variabile x e sviluppiamo al secondo ordine. Con questo procedimento si arriva alla espressione Gaussiana della lezione sopra citata.\\
Supponiamo per semplicità di metterci nella situazione $\overline{x} = 0 $, in questo modo il valor medio di $x^2$ sarà già il valor medio della fluttuazione:
\[
	\overline{x^2}= \sqrt{\frac{B}{2\pi}} \int_{-\infty}^{\infty} x^2 \exp\left( -\frac{1}{2}\beta x^2 \right) dx = \frac{1}{\beta}
.\] 
Questa è la trattazione da cui partiamo, vediamo se si riesce a dare una formulazione generale della ampiezza delle fluttzioni che ci permetta anche di individuare quali fluttuazioni sono correlate l'una all'altra e quali invece sono indipendenti.\\
Partiamo dalla probabilità di avere la fluttuazione \footnote{Adesso la fluttuazione è proprio $\Delta x = x - \overline{x} = x$}: 
\[
	W\left( x \right) \propto \exp\left( \frac{\Delta S_{t}\left( x \right) }{k} \right) 
.\] 
Dove $\Delta S_{t}\left( x \right) = S_{t}\left( x \right) - S_{t}\left( 0 \right) $, con $S_{t} = S + S'$ entropia dell'universo. \\
La variazione di entropia del bagno termico può essere calcolata, visto che lui resta sempre all'equilibrio nelle nostre approssimazioni. Questa sarà:
\[
	\Delta S ' = \frac{\Delta E' + P \Delta V'}{T} = 
	- \frac{\Delta E + P \Delta V}{T}
.\]
Questo perchè l'universo in considerazione è isolato, quindi le quantità complessive del sistemino con il bagno termico si devono conservare. Avendo l'entropia del bagno termico possiamo calcolare l'entropia dell'universo: $\Delta S_{t} = \Delta S + \Delta S'$\\
Quindi la probabilità di avere questa fluttuazione di energia e volume sarà proporzionale a:
\[
	W \ \alpha \ \exp\left( - \frac{\Delta E -T \Delta S + P \Delta V}{kT} \right) 
.\] 
Visto che $W$ ha valore massimo nella posizione di equilibrio possiamo sviluppare $\Delta E$ al secondo ordine attorno al valore più probabile :
\[
	\Delta E = \frac{\partial E}{\partial S} \Delta S  
	+ \frac{\partial E}{\partial V} \Delta V 
	+ \frac{1}{2} \frac{\partial ^2 E}{\partial S^2} \left( \Delta S \right) ^2 
	+ \frac{\partial ^2 E}{\partial S \partial V} \Delta S \Delta V
	+  \frac{1}{2} \frac{\partial ^2E}{\partial V^2} \left( \Delta V \right) ^2 
.\] 
Ci sono molti termini che possono essere riscritti come variabili termodinamiche: 
\[
	\frac{\partial E}{\partial S} = T
.\] 
\[
	\frac{\partial E}{\partial V} = -P
.\] 
Di conseguenza nell'espansione dell'esponenziale i termini al primo ordine vanno via e rimane solo la dipendenza dal secondo ordine:
\begin{align}
	\Delta E - T \Delta S + P \Delta V = &
	\frac{1}{2}\left[
		\frac{\partial ^2 E }{\partial S^2} \left( \Delta S \right) ^2 +
		2 \frac{\partial ^2 E}{\partial S \partial V} \Delta S \Delta V  +
		\frac{\partial ^2 E}{\partial V^2} \left( \Delta V \right) ^2
	\right] =\\
	=&\frac{1}{2}\left[ 
		\Delta S \left( 
			\frac{\partial ^2E}{\partial S^2} \Delta S +
			\frac{\partial ^2E}{\partial S \partial V}\Delta V  
		\right) 
		+ \Delta V \left( 
			\frac{\partial ^2 E}{\partial S \partial V} \Delta S + 
			\frac{\partial ^2 E }{\partial V^2} \Delta V 
		\right) 
	\right] 
.\end{align}
Il termine nella prima parentesi tonda è la variazione della derivata di $E$ rispetto a  $S$, Quello nella seconda parentesi tonda invece è la variazione della derivata di E rispetto a V, allora rimane:
\[
	\Delta E - T \Delta S + P \Delta V = 
	\Delta S \Delta\left( \frac{\partial E}{\partial S}  \right) + \Delta V \Delta \left( \frac{\partial E}{\partial V}  \right) =
	\Delta S \Delta T - \Delta V \Delta P
.\] 
Reinserendo nell'esponenziale per $W$ si ha:
\[
	W \ \alpha \ \exp\left( -\frac{\Delta T \Delta S - \Delta P \Delta V}{2kT} \right) 
.\] 
Abbiamo ottenuto una espressione di quattro variabili termodinamiche, tuttavia noi sappiamo che ne bastano due per descrivere l'intero sistema. Usando le relazioni di Maxwell o i calori specifici possiamo trovare ad esempio quanto valgono $\Delta P$ e $\Delta S$ in funzione di $\Delta T$ e $\Delta V$:
\[
	\Delta T \Delta S - \Delta P \Delta V = 
	\frac{c_{V}}{T}\left( \Delta T \right) ^2 + \frac{1}{Vk_{T}}\left( \Delta V \right) ^2
.\] 
Quindi la probabilità di avere una fluttuazione $\Delta T$ e $\Delta V$ è proporzionale a: 
\[
	W\left( \Delta T, \Delta V \right) \propto \exp\left[ -\frac{c_{V} \left( \Delta T \right)^2 }{2kT^2}- \frac{\left( \Delta V \right) ^2}{2kTVk_{T}} \right]
.\] 
Abbiamo il prodotto di due Gaussiane indipendenti! Quindi possiamo fattorizzare la distribuzione nel prodotto di due distribuzioni, qundi facendo il calcolo del valor medio un pezzo diventa unitario, l'altro invece da il valore cercato:
\[
	\overline{\left( \Delta T \right) ^2} = \frac{kT^2}{c_{V}}
.\] 
\[
	\overline{\left( \Delta V \right) ^2} = kTVk_{T}
.\] 
Inoltre abbiamo anche l'informazione che le fluttuazioni su $T$ e $V$ sono scorrelate \footnote{Nel sistema in cui può variare tutto.}:
\[
	\overline{\left( \Delta T \right) \left( \Delta V \right) }= 0
.\] 
In modo analogo, se avessimo usato altre relazioni avremmo ottenuto che $P$ ed $S$ sono indipendenti.
\[
	\overline{\left( \Delta S \right) ^2} = kC_{P}
.\] 
\[
	\overline{\left( \Delta P \right) ^2} = - \frac{kT}{V}k_{S}
.\] 
Tipicamente variabili intensive ed estensive sono indipendenti l'una dall'altra.
\subsection{Indeterminazione dell'entropia}%
Cerchiamo di stimare le indeterminazioni \ldots dal punto di vista di un qualche parametro fisico del sistema.\\
Preso un sistema isolato con energia $\mathcal{E} _0$ avente un'incertezza su tale energia $\delta \mathcal{E} _0$, ci aspettiamo che la densità di stati $\rho ( \mathcal{E} ) $ sia monotona crescente nella energia \footnote{Più energia abbiamo nel sistema più abbiamo modi per distribuirla tra le varie particelle e quindi aumenta il numero di stati possibili e di conseguenza la densità di stati. Che succede se invece la densità di stati decresce con l'energia? (domanda di esame). Servirebbe un sistema con dei "Gap" tali di energia per cui, all'interno del Gap, $\rho ( \mathcal{E} ) $ diminuisce.}. \\
Se tale funzione è crescente in $\mathcal{E} $ allora possiamo scrivere:
\[
	\rho ( \mathcal{E}_{0}) \ge \frac{1}{\mathcal{E} _{0}} \int_{0}^{\mathcal{E}_{0}} \rho( \mathcal{E} ) d\mathcal{E}   
.\] 
Visto che il numero di stati nella shell spessa $\delta \mathcal{E} _{0}$ è $\rho ( \mathcal{E} _{0}) \delta \mathcal{E} _{0}$ allora l'entropia in tale regione sarà data da:
\[
	S = k\ln\left[ \rho ( \mathcal{E} _{0}) \delta \mathcal{E} _{0} \right] \ge
	k\ln \frac{\delta \mathcal{E} _{0}}{\mathcal{E} _{0}} + k\ln \int_{0}^{\mathcal{E} _{0}}\rho ( \mathcal{E} ) d\mathcal{E} 
.\] 
ci aspettiamo inoltre che
\[
	\int_{0}^{\mathcal{E} _{0}}  \rho ( \mathcal{E} ) d\mathcal{E} \ge \rho ( \mathcal{E} _{0}) \delta \mathcal{E} _{0}
.\] 
Ovvero che il volume della sfera che sta tra $0$ ed $\mathcal{E} _{0}$ sia maggiore del volumino della sola shell al bordo, per verificare questa basta prendere $\mathcal{E} _{0}$ sufficientemente grande.\\
Ci aspettiamo quindi che l'entropia sia limitata dalle precedenti disuguaglianze:
\[
	k\ln\left[ \int_{0}^{\mathcal{E} _{0}} \rho ( \mathcal{E} ) d\mathcal{E} - \ln \frac{\mathcal{E} _{0}}{\delta \mathcal{E} _{0}} \right] 
	\le S 
	\le k \ln\left[ \int_{0}^{\mathcal{E} _{0}} \rho ( \mathcal{E} ) d\mathcal{E}   \right] 
.\] 
allora l'incertezza sull'entropia dovuta all'incertezza $\delta \mathcal{E} _{0}$ sull'energia è:
\[
	\delta  S \sim k\ln \frac{\mathcal{E} _{0}}{\delta \mathcal{E} _{0}}
.\] 
Abbiamo un aspetto poco intuitivo: diminuendo l'incertezza sull'energia $\delta  \mathcal{E}_{0}$ aumenta l'incertezza sull'entropia. Questo è vero per piccole incertezze:
\[
	\frac{\mathcal{E} _{0}}{\delta  \mathcal{E} _{0}} \ll \exp\left( \frac{S}{k} \right)  
.\]
dove essenzialmente tale incertezza è un effetto di bordo. 
L'incertezza al bordo della superficie introduce una indeterminazione nel conteggio sul numero di celle che vengono attraversate dalla nostra superficie, quindi l'errore nel conteggio diventa più pesante mano a mano che restringiamo l'incertezza sull'energia. 
Siccome l'entropia è in un certo senso "l'addetta" a tale conteggio, aumenterà il suo errore restringendo lo spessore del bordo.\\
\subsection{Tempo di vita media degli stati}%
Cerchiamo di stimare $\delta \mathcal{E} _{0}$ legandola ad un parametro fisico, per far questo possiamo usufruire del principio di indeterminazione:
\[
	\delta \mathcal{E} _{0} \tau  \sim \hbar
.\] 
$\tau $ è il tempo di vita media degli stati all'interno del nostro sistema.
Quindi abbiamo dalla relazione scritta sopra:
\[
	\tau \ll \frac{\hbar}{\mathcal{E}_{0} }\exp\left( \frac{S}{k} \right) 
.\] 
Quindi il tempo di vita media degli stati non può essere infinito.\\
Tuttavia l'ultima non è una restrizione molto stringente, infatti in genere si ha che $S \gg k$, quindi l'esponenziale è un numero enorme. \\
Di conseguenza possiamo assumere che i tempi di vita degli stati termodinamici da noi considerati non siano infiniti ma comunque ragionevolmente lunghi. \\
D'altra parte noi ci aspettiamo che l'incertezza quantistica debba essere molto minore dell'incertezza termodinamica del nostro sistema, altrimenti tutte le considerazioni fatte sulla forma delle distribuzioni andrebbero completamente rivisitate (se non demolite). Quindi facendo le misure non saremo in grado di vedere $\delta \mathcal{E} _{0}$, si avrà:
\[
	\delta \mathcal{E} _{0}\ll \sqrt{\overline{\left( \Delta E \right) ^2}} = \sqrt{k C_{V}} T
.\] 
Quindi abbiamo gli estremi del tempo di vita medio $\tau $:
\[
	\frac{\hbar}{T\sqrt{k C_{V}} } \ll \tau \ll \frac{\hbar}{\mathcal{E}_{0} } \exp\left( \frac{S}{K} \right) 
.\] 
Anche quest'ultima non è particolarmente stringente, noi opereremo in sistemi aventi sempre tempi di vita così vincolati.\\

