\clearpage 
\section*{Domande di esame per il capitolo}%
\begin{quest}[Cosa significa misurare il fattore di struttura?]{quest:Cosa significa misurare il fattore di struttura}
    Il fattore di struttura è il risultato di uno scattering elastico di onde su un bersaglio fisso che può essere un qualunque tipo di materiale contenente dei centri diffusori. L'emissione di tale materiale forma una figura di interferenza (su un rivelatore da descrivere con il classico disegnino) che ci da informazioni sulla struttura del materiale. 
    La controparte matematica della figura di scattering elastico è appunto il fattore di struttura:
    \[
	S(\v{q}) = \frac{1}{N}\left<\sum_{i,j}^{} \exp\left(-i\v{q}\left(\v{r}_i-\v{r}_j\right)\right)\right> - N\delta_{0,\v{q}}
    .\] 
    Dove $\v{q}=\Delta\v{k}$ corrisponde alla differenza tra il vettore d'onda uscente ed il vettore d'onda entrante.
    Misurando questa quantità si indaga lo spazio in trasformata di Fourier del reticolo, infatti questa quantità ha un legame di trasformata con la funzione di correlazione di coppia $g(r)$.
\end{quest}
\begin{quest}[Cosa si deve utilizzare per modellizzare lo scattering inelastico?]{quest:Cosa si deve utilizzare per modellizzare lo scattering inelastico?}
    È necessario utilizzare il fattore di struttura dinamico $S(k,\omega)$, anche in questo caso ci sarà una funzione di correlazione di coppia dinamica $G(r,t)$ del tutto analoga alla $g(r)$ (adesso si deve fare la trasformata in $k$ e $\omega$) , la prima indica la probabilità di trovare un'altra particella a distanza $r$ ed al tempo $t$.
\end{quest}
\begin{quest}[A cosa serve la funzione di scattering inelastica?]{quest:A cosa serve la funzione di scattering inelastica?}
Principalmente la si utilizza per trovare la legge di dispersione dei fononi.
\end{quest}
\begin{quest}[Esistono dei fononi nei liquidi? Come potrei definirli?]{quest:Esistono dei fononi nei liquidi? Come potrei definirli?}
Possiamo definire dei fononi nei liquidi come onde di pressione oppure come delle fluttuazioni di densità.
\end{quest}
\begin{quest}[Forma delle bande di energia per i sistemi cristallini]{quest:Forma delle bande di energia per i sistemi cristallini}
Buon disegno.
\end{quest}
\begin{quest}[Qual'è la differenza tra fononi ottici ed acustici?]{quest:Qual'è la differenza tra fononi ottici ed acustici?}
\end{quest}
\begin{quest}[Come deve essere fatto un crstallo per supportare fononi ott.?]{quest:Come deve essere fatto un crstallo per supportare fononi ott.?}
\end{quest}


