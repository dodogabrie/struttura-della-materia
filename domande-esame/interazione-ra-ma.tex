\clearpage 
\section*{Domande di esame per il capitolo}%
\begin{quest}[Parlare dei laser in generale]{quest:Parlare dei laser in generale}
\end{quest}
\begin{quest}[Cosa si intende per pompaggio attivo?]{quest:Cosa si intende per pompaggio attivo?}
\end{quest}
\begin{quest}[Modello di Einstein per l'interazione radiazione materia]{quest:Modello di Einstein per l'interazione radiazione materia}
Einstein impose che la probabilità di transizione da un livello $m$ ad un livello $n$ a causa della presenza del campo EM fosse:
\[
    P_{m\to n} = u(\nu_{nm}) B_{m\to n}
.\] 
Nella quale $B_{m\to n}$ è il coefficiente di Einstein che regola l'emissione spontanea (e l'assorbimento, a meno di un eventuale fattore).\\
Il problema con questa trattazione deriva dal principio del bilancio dettagliato, questo afferma che:
\[
    P_{12}N_1=P_{21}N_2
.\] 
Se fossero semplicemente $P_{12}=P_{12}$  (quindi se tra emissione ed assorbim. ci fosse completa simmetria) allora si avrebbe che:
\[
    \frac{N_2}{N_1}=1
.\] 
Tuttavia per particelle classiche ad esempio sappiamo che:
\[
    \frac{N_2}{N_1} = \exp\left(-\frac{E_{21}}{kT}\right)
.\] 
Questo conflitto venne risolto da Einstein introducendo il coefficiente di emissione spontanea $A_{12}$, infatti scrisse che la probabilità di emettere, quindi di andare da $1$ a $2$ è:
\[
    P_{12}=B_{12}u(\nu_{12}) + A_{12}
.\] 
Mentre quella di assorbimento:
\[
    P_{21} = B_{21}u(\nu_{21}) 
.\] 
Quindi sfruttando Boltzmann e la distribuzione di Plank (per la $u$)
\[
    u(\nu_{12}) d\nu_{12}  = \frac{8\pi\nu_{12}^3}{c^3}\frac{1}{\exp\left(\frac{h\nu_{12}}{k_BT}\right)-1}
.\]  
si trova come deve essere fatto $A$, inoltre grazie al fatto che tale coefficiente deve essere indipendente dalla temperatura si ottiene anche che:
\[
    B_{12}=B_{21}
.\] 
E con le degenerazioni si ha che:
\[
    B_{12}g_1= B_{21}g_2
.\] 
\end{quest}
\begin{quest}[La trattazione per ricavare $A$ di Einstein vale per particelle Quantistiche?]{quest:La trattazione per ricavare $A$ di Einstein vale per particelle Quantistiche?}
In tal caso $\frac{N_1}{N_2}$ è diverso da quello dato da Boltzmann, tuttavia i coefficienti di Einstein continuano a funzionare.
Questo perché si modificano le equazioni di Rate, infatti i numeri di occupazione sono fortemente diversi nel caso quantistico rispetto a quello classico. Nel caso di fermioni la probabilità di raggiungere lo stato finale è proporzionale sia al numero di particelle dello stato iniziale $n_i$ ma anche a $1-n_f$ ($n_f$ numero di particelle nello stato finale). Nel caso di bosoni è proporzionale anche a $1+n_f$.
\end{quest}
\begin{quest}[Motivi per cui si ha una larghezza di riga]{quest:Motivi per cui si ha una larghezza di riga}
    Ha praticamente chiesto l'ultimo paragrafo, da sapere tutto per rispondere\ldots
\end{quest}
\begin{quest}[Chi determina la freq. di emissione di un LASER e chi la larg. di riga]{quest:Chi determina la freq. di emissione di un LASER e chi la larg. di riga}
La della cavità è data dal fattore di Forma $Q$
\[
    \Delta\omega  = \frac{1}{\tau}=\frac{\omega}{Q}
.\] 
mentre la frequenza è data da dalla cavità che avrà dei modi normali dovuti alla fase introdotta.
\end{quest}
\begin{quest}[Cosa determina la larghezza di riga dell'emissione del LASER?]{quest:Cosa determina la larghezza di riga dell'emissione del LASER?}
La larghezza residua del laser è dovuta alla emissione spontanea.	
\end{quest}
\begin{quest}[Perché la larghezza del LASER è molto più piccola di quella della cavità?]{quest:Perché la larghezza del LASER è molto più piccola di quella della cavità?}
Perché il mezzo attivo funziona per emissione stimolata e l'emissione stimolata sputa sempre fotoni tutti uguali: il primo fotone che fa partire questa emissione prende il sopravvento.
\end{quest}
\begin{quest}[Aumentando il pompagg. che succede se ho un allargam. omog. o disomog.?]{quest:Aumentando il pompagg. che succede se ho un allargam. omog. o disomog.?}
Se ho un allargamento omogeneo non cambia nulla, se è disomogeneo ho che ogni specie atomica contribuisce ad una frequenza diversa. Se nel secondo caso aumento il pompaggio potrei avere altre popolazioni che arrivano a stazionarietà ed iniziano a emettere. Nei sistemi disomog. i laser funzionano si più modi contemporaneamente.
\end{quest}
\begin{quest}[Come cambia il rapporto tra E. Stimolata ed E. Spont. se i fotoni sono Fermioni?]{quest:Come cambia il rapporto tra E. Stimolata ed E. Spont. se i fotoni sono Fermioni?}
Nel caso di Fotoni che conosciamo noi si ha che:
\[
    \frac{B_{12}u(\nu_{12})}{A_{12}} = \frac{1}{\exp\left(\frac{\hbar \nu_{12}}{kT}\right)-1}
.\] 
Quindi la probabilità di transire si scrive anche:
\[
    P_{12} = B_{12}u(\nu_{12}) + A_{12} = A_{12}\left(\left<n\right> + 1\right)
.\] 
In particolare si ha che: 
\[
    B_{12}u = A_{12}\left<n\right> = B_{21}u
.\] 
Se i fotoni fossero Fermioni dobbiamo rivedere alcune cose, possiamo intanto assumere che l'emissione segua la seguente legge:
\[
    P_{12} = A\left(1+f\left(\left<n\right>\right)\right)
.\] 
Dove $f$ è una funzione di $\left<n\right>$ che ci dobbiamo ricavare. Supponiamo che all'equilibrio termodinamico la prob. di assorbimento sia la stessa (soltanto che adesso $\left<n\right>$ segue la distribuzione di Fermi):
\[
    P_{21} = A\left<n\right>
.\] 
Imponiamo adesso l'equilibrio termodinamico nuovamente:
\[
    \frac{P_{12}}{P_{21}} = \exp\left( -\frac{h\nu_{12}}{kT}\right) = \frac{1+f\left(\left<n\right>\right)}{\left<n\right>}
.\] 
(Nell'orale è stato messo il $+$ in questa formula, nonostante non sia corretto esprimere la Boltzmann con il $+$ all'esponente in questo caso)\ldots\\
Dove $\left<n\right>$ è:
\[
    \left<n\right> = \frac{1}{\exp\left(\frac{h\nu_{12}}{kT}\right)+1}
.\] 
In questo modo si arriva a 
\[
    f\left(\left<n\right>\right)= -\left<n\right>
.\] 
Quindi abbiamo per fermioni che:
\[
    P_{12} = A\left(1-\left<n\right>\right)
.\] 
Come ci si aspetta, se c'è una particella sotto non si emette.
\end{quest}

