\clearpage
\section*{Domande di esame per il capitolo}%
\begin{quest}[Andamento della $\rho (\mathcal{E})$ in due dimensioni]{quest:Andamento della in due dimensioni}
    $\rho (\mathcal{E}) _\text{2D} \propto  \mathcal{E}^0 = \text{cost}$.
\end{quest}
\begin{quest}[Perché nel caso della demagn. adiab. $S\to 0$ con $T\to 0$?]{quest:Perché nel caso della demagn. adiab. }
Per via del terzo principio della termodinamica.
\end{quest}
\begin{quest}[Qual'è il limite dell'entropia ad alta temperatura?]{quest:Qual'è il limite dell'entropia ad alta temperatura?}
Per $\mu B /kT \ll 1$ abbiamo che $Z_1\to 2$, quindi $\Gamma  = Z = 2^N$, l'entropia sarà:
\[
    S = N k\ln 2
.\] 
È importante ricordare che questa entropia sarà funzione di $\mu B /kT$.
\end{quest}
\begin{quest}[Funzionamento della dem. adiabatica]{quest:Funzionamento della dem. adiabatica}
    Il prof è particolarmente affezionato a questa tecnica, è ben spiegata nel Feynmann di fisica II.
\end{quest}




