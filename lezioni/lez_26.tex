\lez{26}{23-04-2020}{}
Abbiamo assunto nella trattazione che le popolazioni dei livelli $\ket{1}$ e $\ket{2}$ rimanessero invariate incidendo con della radiazione sul sistema, in generale questa assunzione non è rispettata.\\ 
Può infatti capitare di avere un oggetto completamente passivo che viene fatto attraversare dalla radiazione, ci aspettiamo che aumentando l'intensità della radiazione il numero di atomi eccitati da quest'ultima cresca: 
\[
    I\nearrow\implies N_1\nearrow
.\] 
Più atomi si eccitano e meno probabilità avremo di fare assorbimento.
Ci saranno poi dei processi che tengono a riportare il sistema in una condizione di equilibrio termodinamico (emissione spontanea, collisioni con i fononi o con le pareti del contenitore).\\
Cerchiamo di modellizzare questo sistema, vedremo che il coefficiente di assorbimento in questo caso dipenderà dalla intensità.\\
Il coefficiente trovato nella scorsa lezione è:
\[
    k = \frac{h\nu}{c}g(\nu-\nu_{12})B \left(N_2-N_1\right)
.\] 
Dove $g(\nu-\nu_{12})$ è la forma di riga della transizione. Ricordiamo anche che il livello superiore è (1) mentre il livello inferiore è il (2) e che le popolazioni sono definite per unità di volume.
Oppure se la degenerazione dei livelli è diversa abbiamo:
\[
    k = \frac{h\nu}{c}g(\nu-\nu_{12}) \left(B_{21}N_2-B_{12}N_1\right)
.\] 
Descriviamo un sistema che abbia popolazione dei livelli variabile:
\[\begin{aligned}
    \frac{\text{d} N_1}{\text{d} t} =& B_{21}N_2g(\nu-\nu_{12}) u 
    -B_{12}N_1 u g(\nu-\nu_{12}) =\\
	&=\left(B_{21}N_2- B_{12}N_1\right)ug(\nu-\nu_{12}) 
.\end{aligned}\]
Ovviamente per l'altro livello abbiamo la stessa cosa con il segno opposto:
\[
    \frac{\text{d} N_2}{\text{d} t} = -\frac{\text{d} N_1}{\text{d} t}
.\] 
La differenza delle ultime due equazioni ci da:
\[
    \frac{\text{d} \left(N_2-N_1\right)}{\text{d} t} =
    2\left|B_{12}N_1-B_{21}N_2\right|u g(\nu-\nu_{12}) 
.\] 
Questa riguarda l'interazione con il campo elettromagnetico, dobbiamo tener di conto anche dei processi che riportano il sistema all'equilibrio: introduciamo quindi il parametro $\gamma_{\parallel}$.
\begin{defn}[Rate di ritorno all'equilibrio]{defn:Rate di ritorno all'equilibrio}
Definiamo il rate di ritorno all'equilibrio come $\gamma_\parallel$, questo termine è un termine Euristico che ci descrive il ritorno all'equilibrio nella equazione per $N_2-N_1$. 
\end{defn}
Descriviamo la configurazione di equilibrio con le popolazioni $N_1^0$ e $N_2^0$ che seguono la distribuzione di Boltzmann:
\[
    \frac{N_2^0}{N_1^0} = \frac{g_2}{g_1}e^{-\left(E_2-E_1\right) /kT}
.\] 
Supponendo che questo nuovo rate descriva dei processi che tendono a far tornare il sistema all'equilibrio in modo esponenziale aggiungiamo un termine alla equazione del trasporto (come abbiamo fatto per l'equazione del trasporto) del tipo:
\[
    \frac{\text{d} \left(N_2-N_1\right)}{\text{d} t} =
    -2\left|B_{21}N_2-B_{12}N_2\right|u g(\nu-\nu_{12}) 
    +\gamma_{\parallel}\left[\left(N_2^0-N_1^0\right)-
    \left(N_2-N_1\right)\right]
.\] 
Il termine aggiunto è un termine lineare in tale equazione che tiene di conto della differenza tra la situazione di equilibrio e quella fuori equilibrio che consideriamo.\\
Possiamo quindi riscrivere l'equazione di Rate assumendo anche $B_{12}=B_{21}$:
\[
    \frac{\text{d} \left(N_2-N_1\right)}{\text{d} t} =
    \gamma_{\parallel}\left[N_2^0-N_1^0 - \left(N_2-N_1\right)\right]
    -2g(\nu-\nu_{12}) \frac{I}{c}B_{21}\left(N_2-N_1\right)
.\] 
Introducendo questo parametro di rate adesso possiamo di nuovo assumere che la derivata sia nulla (soluzioni stazionarie) quindi possiamo ricavare che:
\[
    N_2-N_1 = \frac{N_2^0-N_1^0}{1+2g(\nu-\nu_{12})
    \frac{B}{\gamma_{\parallel}c}I}=
    \frac{N_2^0-N_1^0}{1+\frac{I}{I_s(\nu)}}
.\] 
In cui per compattare l'espressione si introduce:
\begin{defn}[Intensità di dimezzamento]{defn:Intensità di dimezzamento}
L'intensità per la quale la differenza delle popolazioni si dimezza rispetto al valore di equilibrio è:
\[
    I_s = \frac{\gamma_\parallel c}{2B_{21}g(\nu-\nu_{12})}
.\] 
\end{defn}
All'aumentare dell'intensità la differenza di popolazioni tra il fondamentale ed il livello di sopra diventa più piccolo, questo perché a stazionarietà ho sempre più processi che tengono ad eccitare gli atomi quindi il $\gamma_\parallel$  non riesce a riportare il sistema esattamente all'equilibrio. \\
Se l'intensità tende a $\infty$ allora i due livelli tendono ad avere la stessa popolazione ($N_2-N_1\to 0$)
\footnote{Questo indipendentemente dalle dimensioni di $\gamma_\parallel$.}.
\begin{fact}[Con due livelli niente inversione]{fact:Con due livelli niente inversione}
Per un sistema a due livelli a stazionarietà non è possibile fare inversione di popolazione inviando una radiazione elettromagnetica, al massimo si riesce a fare $N_2=N_1$.
\end{fact}
Dobbiamo notare che questo approccio non contraddice le oscillazioni di Rabi, in questo caso siamo a stazionarietà mentre con tali oscillazioni non siamo stazionari. \\
Possiamo riscrivere il coefficiente di assorbimento inserendo il nuovo valore di $N_2-N_1$:
\[
    k= \frac{h\nu}{c}g(\nu-\nu_{12})B_{21} \frac{N_2^0-N_1^0}{1+\frac{I}{I_s}}=
    \frac{k_0}{1+\frac{I}{I_S}}
.\]
\begin{figure}[H]
    \centering
    \begin{tikzpicture}
	\begin{axis}[
	    xmin= 0, xmax= 11,
	    ymin= 0, ymax = 11,
	    xlabel={$I$},
	    ylabel={$k$},
	    axis lines = middle,
	    ytick={10},
	    yticklabel={$k_0$},
	    xtick={0}
	]
	\addplot[domain=0:10, samples=100]{10*1/(1+x)};
	\end{axis}
    \end{tikzpicture}
    \caption{Andamento di $k$ al variare della intensità.}
    \label{Andamento di k0 al variare della intensità.}
\end{figure}
\noindent
Questo fenomeno si chiama saturazione: aumentando l'intensità esauriscono gli atomi nel fondamentale. L'equazione di propagazione risulta quindi adesso modificata, la soluzione non ci porta più ad un esponenziale:
\[
    \text{d}I=-\frac{k_0}{1+\frac{I}{I_s}}I \text{d}z
.\] 
Il coefficiente di assorbimento è molto utilizzato da chi studia solidi, chi invece studia atomi o impurezze usa la sezione d'urto. Quest'ultima è in relazione al coefficiente di assorbimento tramite la seguente:
\[
    k = \sigma\left(\frac{B_{21}}{B_{12}}N_2-N_1\right)
.\] 
Possiamo anche riscrivere la $\sigma_0$:
\[
    \sigma_0=\frac{h\nu}{c}Bg(\nu-\nu_{12}) =
    \left[\frac{\lambda_{12}}{2}\right]^2Ag(\nu-\nu_{12}) 
.\] 
Dove $\lambda_{12}$ è la lunghezza d'onda della transizione: 
\[
    \lambda_{12} = \frac{c}{\nu_{12}}
.\] 
Siamo interessati a sistemi in cui $N_1>N_2$ in modo tale che il nostro mezzo sia in grado di amplificare la radiazione (coefficiente di assorbimento negativo). Possiamo far questo senza usare una eccitazione ottica (utilizzando ad esempio una eccitazione elettrica con il trasporto di carica).\\
Tuttavia l'inversione di popolazione storicamente si ottiene tramite sistemi ottici, il modo più semplice per farcela è con un sistema a tre livelli. Tipicamente i sistemi a tre livelli descrivono bene i Maser (Laser per le Microonde) che sono venuti prima dei sistemi ottici.
\subsection{Inversione di popolazione per un sistema a 3 livelli}%
Prendiamo un sistema a 3 livelli
\begin{figure}[ht]
    \centering
    \incfig{sistema-a-3-livelli}
    \caption{Sistema a 3 livelli}
    \label{fig:sistema-a-3-livelli}
\end{figure}
Supponiamo di essere nelle microonde e di avere tutte e due le differenze di energia minori di $kT$:
\[\begin{aligned}
    &h\nu_{12}\ll kT\\
    &h\nu_{23}\ll kT
.\end{aligned}\] 
Andiamo a pompare dall'esterno tra i livelli (3) e (1), in questo modo se la pompa è molto intensa le popolazioni in $\ket{1}$ e $\ket{3}$ possono diventare circa uguali:
\[
    N_1\approx N_3
.\] 
Supponiamo che la popolazione del livello (2) sia legata agli altri due semplicemente dall'equilibrio termodinamico 
\footnote{(2) si accorge della popolazione di (1) e (3) per via dei processi di rilassamento che tendono a riportare il sistema in equilibrio.}.
Possiamo quindi dire che (sviluppando al primo ordine gli esponenziali con $h\nu\ll kT$) :
\[\begin{aligned}
    &N_3^0 
    \approx N_2^0\left(1+ \frac{h\nu_{23}}{kT}\right)\\
    &N_1^0 
    \approx N_2^0\left(1-\frac{h\nu_{12}}{kT}\right)
.\end{aligned}\]
Dove i segni diversi sono dovuti al fatto che nel primo caso i pedici sono stati invertiti.\\
Nella ipotesi in cui $N_1\sim N_3$ possiamo fare una media:
\[
    N_1\sim N_3 = N_2^0\left(1+ h \frac{\nu_{23}-\nu_{12}}{2kT}\right)
.\] 
Se siamo nella ipotesi in cui $\nu_{23}>\nu_{12}$ allora abbiamo che $N_1>N_2$ e si ha inversione di popolazione nel seguente punto:
\begin{figure}[H]
    \centering
    \incfig{inversione-di-popolazione-1-2}
    \caption{Inversione di popolazione 1-2}
    \label{fig:inversione-di-popolazione-1-2}
\end{figure}
\noindent
viceversa se $\nu_{23}<\nu_{12}$ abbiamo che $N_2<N_3$, quindi:
\begin{figure}[H]
    \centering
    \incfig{inversione-di-popolazione-2-3}
    \caption{Inversione di popolazione 2-3}
    \label{fig:inversione-di-popolazione-2-3}
\end{figure}
\noindent
Quindi pompando i livelli (1) e (3) possiamo invertire la popolazione di altri due livelli. 
Con questa ulteriore inversione abbiamo che $k<0$, quindi nel propagarsi della radiazione nel nostro mezzo essa cresce esponenzialmente (almeno fino a che la popolazione non satura nel nuovo livello).
\subsection{Laser}%
Un sistema che inverte la popolazione è alla base della amplificazione della radiazione. Se aggiungiamo a quest'ultimo un oscillatore con feedback possiamo avere un dispositivo auto-oscillante dal momento che il guadagno diventa maggiore od uguale delle perdite.\\
Utilizzando degli specchi è quindi possibile, sfruttando l'inversione di popolazione, creare un sistema auto-oscillante molto selettivo in frequenza: un laser
\footnote{Light Amplification by Stimulated Emission of Radiation} 
(o maser nelle micro-onde). \\ 
Notiamo che si chiama Laser nonostante questo sia solo un oscillatore, non un amplificatore. L'amplificatore è un oggetto identico al laser ma senza gli specchi
\footnote{Il nome corretto sarebbe stato Loser, non era tuttavia un nome carino da dare ad una tecnologia tanto preziosa\ldots}.
Un laser per poter auto-oscillare deve avere un mezzo che guadagna (che abbiamo descritto) e dobbiamo inserire un circuito di Feedback (la cavità).\\
Un esempio di cavità può essere un sistema di 4 specchi
\begin{figure}[H]
    \centering
    \incfig{sistema-di-specchi-per-il-laser}
    \caption{Sistema di specchi per il Laser}
    \label{fig:sistema-di-specchi-per-il-laser}
\end{figure}
\noindent
Uno di questi specchi dovrebbe essere semitrasparente se vogliamo che esca qualcosa da questo dispositivo. Notiamo che è la cavità che determina la frequenza di emissione.\\
C'è poi un ulteriore tipo di cavità ottica che sfrutta soltanto 2 specchi facendo passare la radiazione 2 volte per periodo nell'amplificatore, concentriamoci su questo tipo di cavità.\\
Supponiamo che la cavità abbia volume $V$ (con gli specchi di superficie $S$ ed una distanza $L$ tra questi)  e che all'interno la densità di energia del campo EM sia uniforme, in questo modo l'energia totale all'interno della cavità sarà
\[
W = V\cdot u = S\cdot L\cdot u
.\] 
Questa cavità avrà delle perdite, la potenza persa dalla cavità sarà
\[
P_d+P_u
.\] 
In cui $P_d$  è quella dissipata da aria, dagli specchi imperfetti ecc\ldots mentre $P_u$  è la potenza persa a causa dello specchio semitrasparente, quest'ultima è quella che vogliamo: quella che ci dà la luce.\\
Possiamo inglobare queste informazioni introducendo un tempo medio in cui la radiazione esce dalla cavità $\tau$:
\[
\frac{\text{d} W}{\text{d} t} = - \frac{W}{\tau}
.\] 
Dove possiamo immaginare $\tau$ come il tempo di vita media dei fotoni nella cavità.\\
Assumiamo quindi fenomenologicamente che l'energia nella cavità venga persa in modo esponenziale. Solitamente anziché usare $\tau$  si usa il fattore di qualità della cavità:
\[
Q = \omega\tau
.\] 
Nella quale $\omega$ è la frequenza di risonanza determinata dalla fase (per avere un sistema auto-oscillante bisogna che dopo un giro nella cavità ottica il campo sia in fase per avere interferenza costruttiva).\\
Quindi l'uguaglianza "Guadagno = Perdite" ci dirà la condizione di soglia per la quale il sistema è in grado di auto-oscillare, la frequenza di oscillazione sarà fissata dalla fase che a sua volta sarà determinata dalla lunghezza della cavità.\\
Ribadiamo che $\omega$ è data dai modi risonanti del campo elettromagnetico che esistono nella cavità permessi dalle condizioni al contorno (dipendenti dalla lunghezza).
\subsubsection{Distribuzione Lorentziana della densità di energia della cavità ottica}%
Supponiamo che all'interno della cavità il campo EM sia uniforme:
\[
    E(t) = E(0) \exp\left(-i\omega t- \frac{t}{2\tau}\right)
.\] 
Dove abbiamo inserito un fattore $1 /2$ perché $\tau$ è definito sull'energia che va come il quadrato del campo.\\
A noi interessa la coerenza di questa radiazione, per vederla possiamo effettuare una trasformata di Fourier:
\[\begin{aligned}
    E(\Omega) =& \frac{1}{\sqrt{2\pi} }\int_{0}^{\infty} E(t) e^{i\Omega t}dt=\\
    =&
    \frac{1}{\sqrt{2\pi}}E(0)\int_{0}^{\infty} 
    \exp\left(i\left[\left(\Omega -\omega\right)+ \frac{i}{2\tau}\right]t\right) dt=\\
    =&
    \frac{1}{\sqrt{2\pi}}
    \frac{E(0)}{i\left[\left(\Omega-\omega\right)+\frac{i}{2\tau}\right]}
.\end{aligned}\]
Quindi se prendiamo la $u(\Omega)$:
\[
    u(\Omega) = \frac{\left|E(\Omega)\right|^2}{8\pi} = \frac{1}{16\pi} 
    \frac{\left|E(0)\right|^2}{\left(\Omega-\omega\right)^2+ \frac{1}{4\tau^2}}
.\] 
\begin{figure}[H]
    \centering
    \begin{tikzpicture}
	\begin{axis}[
	    xmin= 0, xmax= 13,
	    ymin= 0, ymax = 10,
	    axis lines = middle,
	    xlabel={$\Omega$},
	    ylabel={$u(\Omega)$},
	    xtick={0,5},
	    ytick={0},
	    xticklabel={$\omega$},
	    %yticklabel={$$},
	]
	\addplot[domain=0+9/10*0:10+9/10*10, samples=100]{1/((x-5)^2+1/9)};
	\addplot +[mark=none,black,dotted] coordinates {(5, 0) (5, 9)};
	\end{axis}
    \end{tikzpicture}
    \caption{Distribuzione in frequenza di $u(\Omega)$.}
    \label{Lorentziana}
\end{figure}
\noindent
Avere un tempo $\tau$ di decadimento dei fotoni dalla cavità si traduce in una distribuzione Lorentziana piccata attorno ai modi risonanti $\omega$ e larga
\[
  \Delta\omega_c = \frac{1}{\tau}=\frac{\omega}{Q}  
.\] 
Di conseguenza possiamo scrivere anche che il fattore di qualità è:
\[
Q = \frac{\omega}{\Delta\omega}
.\] 
Quindi migliori sono gli specchi e più la distribuzione tende ad una $\delta$.
\subsubsection{Esempio: perdite dovute solo alle riflessioni sugli specchi}%
Se ci fossero solo le perdite per riflessione descritte da un coefficiente $R$ (detto riflettività: la capacità di riflettere dello specchio) possiamo chiederci come andrebbe il fattore di qualità. Ipotizziamo di avere uno specchio di area $S$ in una cavità lunga $L$, l'energia che incide per unità di tempo sullo specchio (Potenza incidente) è:
\[
E_\text{specchio} = \frac{Su\text{d}z}{\text{d}t} = Suc
.\] 
L'energia che viene trasmessa dagli specchi dalla riflessione è:
\[
E_\text{Lost} = SucR
.\] 
Quindi il fattore di qualità è:
\[\begin{aligned}
    Q &= \omega\tau = \\
      &= \omega  \frac{W}{P_u} =\\
      &=\omega\frac{SLu}{\left(1-R\right)Scu} =\\
      &=2\pi \frac{L}{\lambda } \frac{1}{1-R}
.\end{aligned}\]
La potenza in uscita dall'oggetto è data da quella incidente meno quella trasmessa dagli specchi (che non esce poiché riflessa) : $W = \left(1-R\right)Scu$.\\
Otteniamo allora i limiti asintodici:
\begin{itemize}
    \item Laser Buono: $R\to 1 \implies  Q\to \infty \implies  \Delta\omega\to 0$, nessuna perdita.
    \item Laser pessimo: $R\to 0 \implies  2\pi  L /\lambda$.
\end{itemize}
Notiamo che in generale il fattore $Q$ è direttamente proporzionale a $L$, quindi una cavità più grande crea una radiazione più coerente.\\
Possiamo capire quest'ultima osservazione considerando che più è lunga la cavità e meno volte la radiazione va a sbattere nello specchio, se la radiazione rimbalza meno volte le perdite saranno minori.\\
Facciamo un esempio numerico:
\begin{itemize}
    \item $L=50$ cm.
    \item $\lambda = 500$nm.
    \item $R = 0.98$. 
\end{itemize}
Con questi valori riusciamo ad ottenere $Q = 3\cdot 10^8$.\\
\subsection{Condizioni di auto-oscillazione per un sistema}%
Vediamo adesso di risolvere il seguente problema, esposto da due punti di vista diversi:
\begin{enumerate}
    \item Data l'inversione di popolazione troviamo $Q$ affinché il sistema oscilli.
    \item Dato $Q$ troviamo l'inversione di popolazione per avere un sistema auto-oscillante.	
\end{enumerate}
Normalmente il modo giusto di porsi la domanda è il secondo, infatti possiamo cambiare $N_1-N_2$ (con pompaggio) mentre $Q$ una volta costruito il Laser è definito (gli specchio ormai gli hai comprati e non puoi scorciare il tubo).\\
La condizione di oscillazione per noi sarà 
\begin{center}
    Guadagno = Perdite
\end{center}
Possiamo iniziare trovando il guadagno del sistema. L'equazione per l'intensità della radiazione che attraversa il mezzo attivo è:
\[
\text{d}I= -kI\text{d}z
.\] 
Senza dimenticare che adesso $k<0$. Moltiplicando tutto per la superficie degli specchi e ricordiamo che $I=cu$:
\[
    \text{d}\left(Scu\right)=\left|k\right|Scu\text{d}z=
    \frac{\hbar \omega}{c}g(\nu-\nu_{12})B_{12}\left(N_1-N_2\right)Scu\text{d}z
.\] 
Ricordiamoci che in questo caso abbiamo $N_1>N_2$. Consideriamo un mezzo attivo di lunghezza $l$ e supponiamo che il guadagno sia ragionevolmente piccolo, in particolare che siamo nel limite in cui $\left|k\right|l\ll 1$, in questo modo invece di inserire nella equazione gli esponenziali possiamo scrivere che dopo aver attraversato il mezzo attivo la variazione di $Scu$ è data da:
\[
    \Delta\left(Scu\right)= \hbar \omega g(\nu-\nu_{12}) B_{12}
    \left(N_1-N_2\right)Scul
.\] 
Possiamo introdurre il fattore di riempimento della cavità (la frazione di cavità che è piena di mezzo attivo) $\eta$:
\[
\eta = \frac{l}{L} 
.\] 
Possiamo quindi riscrivere la variazione di $Scu$:
\[
    \Delta (Scu) = \hbar \omega g(\nu-\nu_{12}) B_{12}\left(N_1-N_2\right)
    \eta W
.\] 
Quindi abbiamo la variazione di potenza dovuta alla amplificazione del mezzo attivo, questa deve eguagliare la variazione di potenza dovuta alle perdite che abbiamo trovato in precedenza (la potenza persa):
\[
P_u = \omega\frac{W}{Q}
.\] 
Uguagliando le due quantità si ha che:
\begin{fact}[Condizione sulla inversione di Pop. per avere auto-oscillazione]{fact:Condizione sulla inversione di Pop. per avere auto-oscillazione}
\[
    N_1-N_2 = \frac{1}{\eta\hbar  g(\nu-\nu_{12})B_{12}Q}
.\]
\end{fact}
Questa ci dice quanto deve essere l'inversione di popolazione a $Q$ fissato per avere una auto-oscillazione del sistema (o se invertiamo per $Q$ abbiamo il viceversa).\\
Possiamo vedere in funzione di $N_1-N_2$ che cosa esce dal Laser (la $P_\text{out}$ ).
\begin{figure}[H]
    \centering
    \incfig{andamento-della-potenza-in-uscita-in-funzione-della-differenza-tra-le-popolazioni}
    \caption{Andamento della Potenza in uscita in funzione della differenza tra le popolazioni}
    \label{fig:andamento-della-potenza-in-uscita-in-funzione-della-differenza-tra-le-popolazioni}
\end{figure}
\noindent
Fino a che siamo al di sotto della condizione di soglia dal laser non esce nulla. Quando arriviamo alla condizione di soglia abbiamo che l'inversione di popolazione non può più crescere perché in tal caso avremo un guadagno più grande delle perdite facendo divergere la $P_\text{out}$.\\
Sopra alla condizione di soglia (G=P) possiamo continuare a pompare il sistema, tutta l'energia che diamo andrà nella emissione stimolata dei fotoni che escono dalla cavità: luce LASER!\\
Non ha quindi senso scrivere la potenza in funzione di $N_1-N_2$, la possiamo invece scrivere in funzione del pompaggio $J$ (Potrebbe essere una densità di corrente oppure un altro laser da qualche parte, insomma una forma di pompaggio). In questo modo l'andamento che si ottiene è il seguente:
\begin{figure}[H]
    \centering
    \begin{tikzpicture}
	\begin{axis}[
	    xmin= 0, xmax= 10,
	    ymin= 0, ymax = 6,
	    axis lines = middle,
	    xlabel={$J$},
	    ylabel={$P_\text{out}$},
	    xtick={0,5},
	    ytick={0},
	    xticklabel={Soglia},
	    yticklabel={},
	]
	    \addplot[domain=0:5, samples=100, red]{0};
	    \addplot[domain=5:9, samples=100, red]{x-5};
	\end{axis}
    \end{tikzpicture}
    \caption{Andamento della potenza uscente in funzione di $J$.}
    \label{Pompaggio}
\end{figure}
\noindent
Una proprietà interessante dei Laser è che i fotoni uscenti sono tutti identici tra di loro, appartengono quindi tutti (circa) alla stessa cella dello spazio delle fasi. \\
Abbiamo trascurato fin'ora l'emissione spontanea, questa nei Laser gioca due importanti ruoli:
\begin{itemize}
    \item Perché si inneschi il processo di oscillazione ci deve essere un fotone, tale rumore è dato dalla emissione spontanea 
    \footnote{Non serve che sia dall'emissione spontanea della zona attiva, può anche esserci un fotone emesso dalla radiazione di corpo nero ad innestare.}.
\item La larghezza di riga, infatti anche se sono pochi i fotoni emessi per emissione spontanea che finiscono esattamente nel modo amplificato ogni tanto qualcuno può finire dentro. Questo causa delle fluttuazioni nelle densità della radiazione emessa e nella frequenza.
\end{itemize}
Quindi il Laser avrà sempre una minima larghezza di riga per via di questi processi, questo è il limite "quantum" della larghezza di riga (larghezza di Schawlow-Townes), tale larghezza è proporzionale a:
\[
    \Delta\omega_\text{ST} \propto\frac{\Delta\omega}{n_\text{phot} } 
.\] 
Con $n_\text{phot}$ il numero di fotoni nella cavità, mentre $\Delta\omega$ è la larghezza di riga della cavità dato dal modello Lorentziano.\\
\begin{fact}[Larghezza di riga di un LASER]{fact:Larghezza di riga di un LASER}
La larghezza in frequenza di un laser tende a stringersi quanto più ci avviciniamo alla soglia (quella trovata per $N_1-N_2$).
\end{fact}
Inoltre non ci scordiamo che noi abbiamo ricavato la larghezza di riga della cavità, che è ben diverso dal ricavare la larghezza di riga del LASER (facciamo un semplice esperimento di trasmissione dalla cavità). La larghezza di riga del laser in funzionamento è moolto più stretta di $\Delta\omega$.
\subsection{Introduzione alla soluzione "Termodinamica+Coerenze" e Matrice densità}%
Abbiamo detto nella scorsa lezione che andando a parlare di popolazioni ci si perde la parte coerente della interazione, ci siamo infatti ristretti a parlare dei moduli quadri delle ampiezze di probabilità perdendo informazioni sulla fase.\\
C'è la necessità di trovare un formalismo che ci permetta di mantenere la descrizione della evoluzione coerente derivante dalla equazione di Schrodinger (che da le oscillazioni di Rabi) e di poter inserire dentro questo formalismo anche la termodinamica (Ensemble di sistemi, popolazioni che seguono l'equilibrio ecc\ldots). \\
Quando si tratta di mischiare termodinamica e MQ tipicamente si usa la matrice densità $\rho$, questo è uno operatore che, scritto su una serie di stati restituisce:
\begin{defn}[Matrice densità]{defn:Matrice densità}
    L'operatore Matrice Densità $\rho$ è così definito:
\[
    \rho_{mn}(t) = \frac{1}{\eta}\sum_{k=1}^{\eta} \left\{a^k_m(t) a^k_n(t) \right\}
.\] 
Nella quale 
\begin{itemize}
    \item $\eta$: Numero di sistemi presenti nella Ensemble (ad esempio il numero di Atomi)
    \item $a_{m}^ka_n^k$: Ampiezze di probabilità per ciascun singolo sistema di trovarsi nello stato $m$ o $n$.
\end{itemize}
\end{defn}
Abbiamo quindi $\eta$ copie dello stesso sistema e ciascuna copia può trovarsi in una qualunque sovrapposizione degli autostati $\varphi_n$ con ampiezza di probabilità $a_n(t)$, la matrice di densità rappresenta una media pesata delle ampiezze di probabilità di trovarsi in ciascuno di questi autostati pesata su tutti gli atomi. \\
Possiamo ricordare alcune proprietà di tale matrice:
\begin{itemize}
    \item $Tr(\rho) = 1$: le ampiezze di probabilità sono normalizzate, gli stati $\ket{\psi_k}$  sono normalizzati, le probabilita $P_k$ sono normalizzate in maniera che la somma delle probabilità faccia 1. 
    \item L'evoluzione della matrice di densità è descritta da:
	\[
	    i\hbar \dot{\hat{\rho}} = \left[\hat{H},\hat{\rho}\right]
	.\] 
	Chiaramente se siamo nelle ipotesi in cui $\dot{\hat{\rho}}=0$ (sistema all'equilibrio) allora significa che l'Ensemble deve essere stazionario e quindi $\hat{\rho}= \hat{\rho}(\hat{H})$.
    \item Il valore di aspettazione di una certa grandezza fisica si può scrivere come:
	\[
	    \left<G\right>= Tr(\hat{\rho}\hat{G}) 
	.\] 
	Per valore di aspettazione si attende valore di aspettazione quantistico (pesato con le ampiezze di probabilità quantistiche) e poi mediato sull'Ensemble.
    \item Gli elementi diagonali della matrice densità si chiamano popolazioni, infatti possiamo scrivere per il livello i-esimo:
	\[
		N_i = N\rho_{ii}
	.\] 
\end{itemize}
Consideriamo una Ensemble di atomi con due livelli (i soliti due), la matrice di densità sarà ovviamente una matrice $2x 2$. 
\begin{defn}[Coerenze]{defn:Coerenze}
Si dicono coerenze elementi fuori diagonale della matrice densità
\[
\rho_{12} \quad \quad \rho_{21}
.\]
\end{defn}
Possiamo vedere il perché di questo nome considerando uno stato puro (in cui non c'è una media quantistica definita e $\rho^2=\rho$), vediamo chiaramente che per avere questi elementi fuori diagonale diversi da zero devono essere diverse da zero entrambe le occupazioni $a_1$ e $a_2$ (perché essenzialmente contengono i complessi coniugati di tali occupazioni).\\
Questa cosa resta vera anche se lo stato non è puro, per avere le coerenze non nulle è necessario che gli $a_1$ e $a_2$ siano entrambi non nulli. Deve essere quindi vero che ogni atomo si trovi in uno stato che è sovrapposizione degli autostati. \\
Quando il sistema si trova in una sovrapposizione quantistica degli autostati allora abbiamo gli effetti di coerenza di oscillazione di evoluzione temporale che ci permettono il passaggio da uno stato all'altro. 
\subsubsection{Esempio di coerenze: operatore di Dipolo}%
Possiamo vedere il ruolo delle coerenze studiando l'operatore di dipolo $\hat{P}$:
\[
    \hat{P}= N\hat{e}p\left(\ket{1}\bra{2}+ \ket{2}\bra{1}\right)
.\]
Dove $\hat{e}$ è il versore del campo elettrico, mentre $p$ è l'elemento di matrice di dipolo:
\[
    p = \bra{1}e\v{r}\cdot \hat{e}\ket{2} \label{eq:mat-dip-el}
.\] 
Che abbiamo supposto essere reale in modo tale da poterlo raccogliere. \\
Facciamo adesso il valor medio dell'operatore di dipolo sull'Ensemble di atomi:
\[
    \left<\hat{P}\right>= Tr\left(\rho\hat{P}\right) =
    Np\hat{e}\left(\rho_{12}+\rho_{21}\right)
.\] 
Quindi il valor medio dell'operatore di dipolo è legato strettamente alle coerenze. Quindi se gli atomi sono in uno stato che è sovrapposizione quantistica degli stati (1) e (2) il dipolo è diverso da zero, altrimenti il dipolo è nullo. \\
Dobbiamo stare attenti al fatto che quello che abbiamo calcolato  è il valore medio del dipolo sull'Ensemble e sugli stati del nostro sistema, non è l'elemento di matrice di equazione \ref{eq:mat-dip-el}.\\
Notiamo che in assenza di campo elettrico il valor medio del nostro dipolo deve essere nullo ($\left<\hat{P}\right>=0$ ), per uno stato puro significa che gli elementi di matrice fuori diagonale sono nulli, quindi significa che $a_1$  oppure $a_2$  e nullo. Tuttavia quest'ultima non può essere una condizione di equilibrio termodinamico (in cui noi vogliamo avere una certa probabilità che il nostro sistema di trovi nello stato (1) ed un'altra probabilità che il sistema stia nello stato (2) con entrambe le prob. non nulle).\\
Quindi uno stato puro non può avere sia dipolo nullo che equilibrio termodinamico, fortunatamente quando siamo in uno stato non puro allora possiamo avere le coerenze nulle ed avere le $\rho_{11}$ e $\rho_{22}$  non nulle (un esempio si trova nell'arimondo).\\
Quindi la $\rho$ ci permetterà di fare medie sull'Ensemble delle popolazioni e al tempo stesso di considerare evoluzione coerente del nostro Ensemble di sistemi sotto l'azione del campo EM.
\subsection{Soluzione della evoluzione temporale per la matrice densità}%
Vorremmo adesso sviluppare questo formalismo utilizzando la matrice densità e mostrando che nei limiti di stati puri troviamo le oscillazioni di Rabi oppure verso il formalismo in cui abbiamo le equazioni di rate con le popolazioni ecc\ldots. In questo modo possiamo capire come si passa da un caso all'altro.\\
Scriviamo allora l'equazione della matrice di densità che descrive l'evoluzione temporale del sistema anche in presenza di un campo EM:
\[
\frac{\text{d} \rho}{\text{d} t} = \frac{1}{i \hbar}\left[H,\rho\right]
.\] 
L'Hamiltoniana sarà come sempre in presenza di un campo elettrico:
\[
H= H_0+V
.\]
dove $H_0$ descrive i livelli energetici:
\[
H_0=E_1\ket{1}\bra{1}+E_2\ket{2}\bra{2}
.\] 
Mentre $V$ è l'Hamiltoniana di interazione con il campo elettromagnetico (facciamo già l'approssimazione di onda rotante che toglie gli elementi di matrice anti-oscillanti):
\[
V=-\frac{1}{2}pE_0e^{-i\omega t}\ket{1}\bra{2}-\frac{1}{2}pE_0e^{i\omega t}\ket{2}\bra{1}
.\] 
A questo punto possiamo riscrivere la derivata della $\rho$ (con il conto già svolto):
\[\begin{aligned}
    &\frac{\text{d} \rho_{11}}{\text{d} t} =
	-\frac{1}{i\hbar }\frac{pE_0}{2}e^{-i\omega t}\rho_{21}
	+\frac{1}{i\hbar }\frac{pE_0}{2}e^{i\omega t}\rho_{12} \\
    &\frac{\text{d} \rho_{22}}{\text{d} t} = 
    -\frac{\text{d} \rho_{11}}{\text{d} t} \\
    &\frac{\text{d} \rho_{12}}{\text{d} t} =
    \frac{1}{i\hbar }\left(E_1-E_2\right)\rho_{12}
    -\frac{1}{i\hbar}p\frac{E_0}{2}e^{-i\omega t}
    \left(\rho_{22}-\rho_{11}\right)
.\end{aligned}\]
Naturalmente si ha che $\rho_{12}=\rho_{21}^*$.
\subsubsection{Vettore di Feynmann e precessione}%
Le equazioni precedenti descrivono in modo completo il nostro sistema, tuttavia per renderle leggibili è necessario riscriverle utilizzando una notazione vettoriale, definiamo:
\begin{defn}[Vettore di Feynmann]{defn:Vettore di Feynmann}
    Il vettore di Feynmann $\v{R}$ è il vettore costruito tramite le componenti della matrice densità nel seguente modo:
\[\begin{aligned}
    &R_1=2Re\left(\rho_{12}e^{i\omega t}\right)=\rho_{12}e^{i\omega t}+ CC\\
    &R_2= -2 Im\left(\rho_{12}e^{i\omega t}\right)=i\rho_{12}e^{i\omega t}- CC\\
    &R_3 = \rho_{11}-\rho_{22}
.\end{aligned}\]
\end{defn}
In questo modo otteniamo delle equazioni:
\[\begin{aligned}
    &\frac{\text{d} R_1}{\text{d} t} = \left(\omega-\omega_{12}\right)R_2\\
    & \frac{\text{d} R_2}{\text{d} t} = -\left(\omega-\omega_{12}\right)R_1+
    p \frac{E_0}{\hbar }R_3\\
    &\frac{\text{d} R_3}{\text{d} t} =-p \frac{E_0}{\hbar }R_2
.\end{aligned}\]
Possiamo usare la notazione vettoriale per compattare ancora di più le equazioni:
\[
    \frac{\text{d} \v{R}}{\text{d} t} = \v{R} \times \v{B} 
    \label{eq:precessione}
.\] 
Dove si introduce il vettore $\v{B}$ :
\[
    \v{B} = \left(\underbrace{p \frac{E_0}{\hbar}}_{\Omega_R} \ , \  0 \,
    ,\ \underbrace{\omega-\omega_{12}}_{\delta}\right)
.\] 
Un vettore la cui equazione del moto è scritta come la \ref{eq:precessione} fa una precessione, il vettore $\v{R}$ precede attorno al vettore $\v{B}$ (la sua derivata è sempre ortogonale a $\v{R}\times \v{B}$), la velocità angolare di precessione è proprio $\left|\v{B}\right|$. \\
Vediamo questa precessione nello spazio $x-y-z$, il vettore $\v{B}$ è così fatto:
\begin{figure}[H]
    \centering
    \incfig{vettore-b-della-precessione}
    \caption{Moto di precessione di $\v{R}$.}
    \label{fig:vettore-b-della-precessione}
\end{figure}
\noindent
Il vettore di Feynmann $\v{R}$ precede attorno alla direzione individuata da $\v{B}$.
\subsubsection{Soluzione alle equazioni con condizioni iniziali: Ritorno delle oscillazioni di Rabi.}%
Supponiamo che all'istante iniziale si accenda il campo EM, in tal caso siamo inizialmente all'equilibrio termodinamico (coerenze nulle $\v{R}_1^0 = \v{R}_2^0=0$) quindi $\v{R}$ è diretto lungo l'asse $z$ all'istante iniziale.
In particolare si ha che:
\[
    R_3^0 = \frac{1}{N}\left(N_1^0 -N_2^0\right)
.\] 
Dove:
\[\begin{aligned}
    &N_1^0 = \frac{N}{Z}e^{-E_1 /kT}\\
    &N_2^0 = \frac{N}{Z}e^{-E_2 /kT}
.\end{aligned}\]
Supponiamo di essere nella situazione più tipica dell'ottica: $E_{12}\gg kT$, in questo caso possiamo assumere che all'equilibrio termodinamico:
\[
N_1^0\sim 0\qquad
N_2^0 \sim N
.\] 
In questo modo abbiamo che:
\[
    \rho_{11}^0\sim 0\qquad
    \rho_{22}^0\sim 1
.\] 
Di conseguenza questa situazione corrisponde ad avere $R_3^0\approx-1$.
Iniziamo allora a capire che aria tira: se $\delta =0$ abbiamo che $\omega = \omega_{12}$: la componente lungo $z$ di $\v{R}$ ruota con velocità angolare $\Omega_R$. 
\begin{figure}[H]
    \centering
    \incfig{condizioni-iniziali-su-r}
    \caption{Condizioni iniziali su R}
    \label{fig:condizioni-iniziali-su-r}
\end{figure}
\noindent
Quindi in particolare dopo un tempo $\pi /\Omega_R$  il vettore $\v{R}$ arriva in $z=+1$ e prosegue con la sua precessione. Queste sono proprio le oscillazioni di Rabi.\\
Ci siamo infatti messi con tutti gli atomi nel fondamentale e abbiamo soltanto l'interazione con il campo EM.\\
Se avessimo $\delta\neq 0$ allora avremmo che $\v{B}$ si inclina e quindi la precessione ci descrive un cerchio nel piano ortogonale alla direzione di $\v{B}$ e smette di arrivare in $z=+1$ esattamente come quando le oscillazioni di Rabi non ci portavano più esattamente ad $z=1$.\\
La frequenza a cui gira in questo caso è: $\sqrt{\Omega_R^2+\delta^2}$. Il formalismo quindi riproduce quindi esattamente le oscillazioni di Rabi.\\
Nella prossima lezione mettiamo la termodinamica: dei processi di rilassamento che ci riportano alla configurazione di equilibrio.

