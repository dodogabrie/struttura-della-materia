\lez{24}{19-04-2020}{}
\subsection{Regola d'oro di Fermi da un sistema perturbato.}%
Riprendiamo la teoria delle perturbazioni (anche dipendenti dal tempo), supponiamo di avere una Hamiltoniana per il nostro sistema imperturbato $H_0$ e di inserire una perturbazione $V(t)$:
\[
    \overline{V}(t) = \begin{cases}
	&W(t)  \quad  0<t<\tau\\
	&0 	\quad \quad t<0 \text{ e } t>\tau
    \end{cases}
.\] 
\begin{figure}[H]
    \centering
    \incfig{perturbazione-a-gradino}
    \caption{Perturbazione dipendente dal tempo.}
    \label{fig:perturbazione-a-gradino}
\end{figure}
\noindent
l'Hamiltoniana completa $H$ sarà data da:
\[
	H=H_0+\overline{V}(t)
.\] 
Cerchiamo le soluzioni alla equazione di Schrodinger dipendente dal tempo:
\[
    i\hbar \frac{\text{d} \psi(t)}{\text{d} t} 
    = \left[H_0+\overline{V}(t) \right]\psi (t) 
.\] 
\subsubsection{Soluzione a livelli discreti}%
Sviluppiamo la funzione d'onda sugli autostati di $H_0$ :
\[
    \ket{\psi(t)}=\sum_{n}^{} a_n(t) e^{-i E_n t /\hbar }\ket{\varphi_n}
.\] 
Ipotizziamo inoltre che per tutti i $t<0$ il sistema si trovi in un autostato $\ket{\varphi_m}$ dato. Vogliamo trovare lo stato del sistema all'istante $ \tau$: quello in cui termina la perturbazione.\\
Per $t\le 0$ abbiamo che 
\[\begin{aligned}
    \begin{cases}
    &\ket{\psi_m}(t) = e^{-iE_m t /\hbar } \ket{\varphi_m}\\
    &a_n(t) = \delta_{mn}
    \end{cases}
.\end{aligned}\]
Al termine della perturbazione mi troverò in uno stato dato dalla sovrapposizione degli altri autostati, tali autostati continueranno ad evolvere (con l'esponenziale della loro energia) e gli $a_n(t)$ per $t\ge \tau$ rimarranno costanti ed uguali a $a(\tau)$.\\
Possiamo chiamare le ampiezze di probabilità $a_n(\tau)$ come $a_{m\to n}(\tau) $ essendo questa l'ampiezza di probabilità dello stato $\varphi_m$ di passare allo stato n-esimo.\\
La probabilità di andare dallo stato $m$ allo stato $n$ sarà data da:
\[
    W_{m\to n}(\tau)=\left|a_{m\to n}(\tau) \right|^2
.\] 
Per trovare tali ampiezze sostituiamo nella equazione di Schrodinger lo sviluppo della $\psi$ e proiettiamo su un generico $\bra{\varphi_n}$.\\
\[
    i\hbar \frac{\text{d} }{\text{d} t} a_{m\to n}(t) 
    = \sum_{l}^{} \bra{n}W(t)\ket{l}e^{i\omega_{n,l} t}a_{m\to l}(t) \label{eq:risposta111}
.\] 
Nella quale $\omega_{n,l}$  vale:
\[
\omega_{n,l}=\frac{E_n-E_l}{\hbar }
.\] 
Ci mettiamo nel caso in cui l'elemento di matrice è nullo tra stati uguali $\bra{n}W\ket{n} = 0$ (lo possiamo sempre fare con una base opportuna degli stati), vedremo che questo è il caso più naturale quando la perturbazione sarà l'interazione elettromagnetica.
Risolviamo queste equazioni con l'ulteriore condizione iniziale che ripetiamo essere:
\[
    a_{m\to n}(t=0) = \delta_{mn}
.\] 
Adesso procediamo con una soluzione iterativa: mettiamo le condizioni iniziali nella parte destra della equazione \ref{eq:risposta111}, otteniamo una espressione per la $a_{m\to n}(t) $ e la reinseriamo nella stessa equazione. \\
In conclusione si ottiene l'equazione di Schrodinger al primo ordine (semplicemente utilizzando la $\delta_{nm}$ delle condizioni iniziali):
\[
    i\hbar \frac{\text{d} a_{m\to n}(t) }{\text{d} t} 
    =
    \bra{n}W(t) \ket{m}e^{i\omega_{nm}t }
.\] 
Possiamo integrare questa espressione (mettiamo un pedice per indicare il primo ordine):
\[
    a_{m\to n}^{(1) } (t) =
    \frac{1}{i\hbar }
    \int_{0}^{t} \bra{n}W(t')\ket{m}e^{i\omega_{mn}t'} dt'
.\] 
Agli ordini successivi abbiamo che
\[
    a_{m\to n}^{(2)}(t)=
    \frac{1}{i\hbar }
    \sum_{m}^{} \int_{0}^{t}  
    \bra{k}W(t')\ket{n}e^{i\omega_{kn}t'}a_{m\to n}^{(1)}(t')  dt'
.\] 
Come primo esempio possiamo prendere la $W(t) = V$  indipendente dal tempo. 
\begin{figure}[H]
    \centering
    \incfig{perturbazione-indipendente-dal-tempo-davvero}
    \caption{Perturbazione indipendente dal tempo.}
    \label{fig:perturbazione-indipendente-dal-tempo-davvero}
\end{figure}
\noindent
In questo caso abbiamo che:
\[
    a_{m\to n}^{(1)}(t) =
    \frac{1}{i\hbar }\bra{n}W \ket{m} \int_{0}^{t} e^{i\omega_{nm}t'}dt'
.\] 
Quindi possiamo trovare la probabilità di transizione da $m$ a $n$ al tempo $\tau$:
\[
    W^{(1)}_{m\to n}(\tau)=
    \left|a_{m\to n}^{(1)}(\tau) \right|^2 = \frac{2}{\hbar^2 }\left|\bra{n}W \ket{m}\right|^2F(\tau,E_n-E_m) 
.\] 
Questa $F$ è:
\[
    F(\tau,x) = \frac{2}{\left(\frac{x}{\hbar }\right)^2}\sin^2\left(\frac{x\tau}{2\hbar }\right) 
.\] 
Questa funzione ha la seguente forma:
\begin{figure}[H]
    \centering
    \incfig{f-interazione-radiazione-materia}
    \caption{Funzione che descrive la probabilità di transizione tra gli stati.}
    \label{fig:f-interazione-radiazione-materia}
\end{figure}
\noindent
In cui il primo massimo si ha per $x=0$ mentre gli zeri di tale funzione si hanno per: 
\[
\left|x\right|=\frac{2\pi\hbar }{\tau}n
.\] 
Maggiore è $\tau$ e più si stringe il massimo della funzione poiché si avvicinano i minimi limitrofi. Se vale
\[
\tau \gg \frac{2\pi\hbar }{\left|x\right|} 
.\] 
Possiamo approssimare tale funzione ad una $\delta$, in tal caso vale la relazione:
\[
    \int_{-\infty}^{\infty} F(x) dx = \tau \hbar \pi 
.\] 
Con questa normalizzazione possiamo scrivere la funzione $F(\tau,x)$: 
\[
    F(\tau,x) =\pi\tau\hbar \delta (x)  
.\] 
Se la perturbazione dura un tempo infinito ($\tau\to \infty$ ) allora l'ultima formula ci dice che il sistema non può cambiare energia. Possiamo vedere questo nella formula di probabilità di transito (in cui vi è la $F$), se questa funzione è una $\delta (E_n-E_m)$ allora il sistema non cambia energia poiché l'unico modo di farlo è avere la stessa energia di prima\ldots\\
\begin{fact}[Perturbazione costante]{fact:Perturbazione costante}
Una perturbazione costante nel tempo conserva l'energia del sistema.
\end{fact}
Questo esprime sia la conservazione dell'energia per il sistema che il principio di indeterminazione (quando l'approssimazione della $\delta$ non è valida), infatti la larghezza che collega le grandezze $x$ e $\tau$ è proprio $\hbar 2\pi$.
\subsubsection{Soluzione a livelli continui.}%
Abbiamo trattato il sistema concentrandoci sulla transizione di due livelli $m$ e $n$, in generale possiamo avere una distribuzione di stati finali continua (come ad esempio la $\rho_e$ nei cristalli) 
\begin{figure}[H]
    \centering
    \incfig{distribuzione-di-stati-finali-continua}
    \caption{Distribuzione di stati finali continua}
    \label{fig:distribuzione-di-stati-finali-continua}
\end{figure}
\noindent
in tal caso possiamo normalizzare:
\[
    \int\rho(E_n) dE_n = 1
.\] 
Possiamo immaginare che questa densità di stati sia distribuita attorno ad una energia centrale con un picco largo $\Delta E_n = \hbar \Delta\omega_n $. 
\begin{figure}[H]
    \centering
    \incfig{distribuzione-degli-stati-continua}
    \caption{Distribuzione degli stati continua}
    \label{fig:distribuzione-degli-stati-continua}
\end{figure}
\noindent
Possiamo calcolare la probabilità di transizione complessiva, ovvero non solo quella di transire dalla energia $m$ a quella $n$ ma quella di passare dall'energia $E_m$ ad un'altra della distribuzione finale:
\[
    \overline{W}^{(1)}_{m\to n}(\tau) =
    \frac{2}{\hbar ^2}\int\left|\bra{n}W \ket{m}\right|^2 F(\tau, E_n-E_m)\rho(E_n) dE_n 
.\] 
Tipicamente nei sistemi che trattiamo il primo termine (che è un modulo quadro di un elemento di matrice) non dipende da quali dei singoli stati $n$ andiamo a considerare, quindi può essere tirato fuori dall'integrale (e ci mettiamo $\overline{n}$ per indicare che non dipende da quale $n$ scegliamo).
\[
    \overline{W}^{(1)}_{m\to n}(\tau) =
    \frac{2}{\hbar ^2}
    \left|\bra{\overline{n}}W \ket{m}\right|^2
    \int
    \frac{2}{\left(\frac{E_n-E_m}{\hbar }\right)^2}
    \sin^2\left(\frac{\left(E_n-E_m\right)\tau}{2\hbar }\right) 
    \rho(E_n) dE_n
.\] 
Nell'ipotesi in cui $\tau\gg 2\pi /\Delta\omega_n$ abbiamo che la densità di stati finali è molto larga rispetto alla durata della interazione. \\
Dal momento che $\tau$ descrive la larghezza della $F$ abbiamo che nello spazio delle frequenze (invertendo la disuguaglianza) $F$ è molto più stretta di $\rho$.
\begin{figure}[H]
    \centering
    \incfig{relazione-tra-f-e-la-densità-di-stati}
    \caption{Relazione tra F e la densità di stati, sulle y abbiamo unità adimensionali.}
    \label{fig:relazione-tra-f-e-la-densità-di-stati}
\end{figure}
\noindent
Con queste considerazioni possiamo portare fuori la $\rho$ dall'integrale, rimane un integrale che si risolve e fa $\pi \tau \hbar $:
\begin{defn}[Regola d'oro di Fermi]{defn:Regola d'oro di Fermi}
    Per un sistema perturbato da una perturbazione di durata $\tau$ molto più grande dell'inverso della larghezza del salto di livelli $\Delta\omega$  la probabilità di passare dal livello $m$ al livello $n$ è:
\[
    W^{(1)}_{m\to n}(\tau) =
    \frac{2\pi}{\hbar}\left|\bra{\overline{n}}W \ket{m}\right|^2\rho(E_{\overline{n}})\tau 
.\] 
Importante notare la linearità della probabilità di transizione dal tempo $\tau$.
\end{defn}
Riassumiamo le ipotesi di applicabilità:
\begin{itemize}
    \item Poter sviluppare perturbativamente al primo ordine ("piccole" perturbazioni da definire dopo).
    \item Durata del tempo di interazione molto maggiore dell'inverso della larghezza della distribuzione degli stati finali.
\end{itemize}
Grazie alla linearità in $\tau$ della regola possiamo definire una probabilità $P_{m\to n}$ per unità di tempo dividendo semplicemente per $\tau$.
\[
    \overline{P}_{m\to n}^{(1) } =
    \frac{2\pi}{\hbar}\left|\bra{\overline{n}}W \ket{m}\right|^2\rho(E_{\overline{n}})
.\] 
Possiamo imporre che tale probabilità prima di essere integrata fosse la $\delta$, quindi:
\[
    P_{m\to n}^{(1) } =
    \frac{2\pi}{\hbar}\left|\bra{n}W \ket{m}\right|^2\delta (E_n-E_m) 
.\] 
Vediamo che in questo modo raggiungiamo sempre gli stati aventi la stessa energia dello stato iniziale:
\[
    \overline{E}_n = E_m
.\] 
Nel caso opposto invece $\tau\ll\frac{2\pi }{\Delta\omega}$, questo è il caso in cui la $F$ è molto larga rispetto alla $\rho$.
In tal caso facciamo il contrario: portiamo fuori la $F$ e resta l'integrale della $\rho$. Facendo in questo modo si ottiene 
\[
\overline{P} \propto \tau^2
.\] 
Non abbiamo più la linearità, non si può definire una probabilità di interazione nell'unità di tempo.
\subsection{Emissione stimolata ed assorbimento}%
Abbiamo parlato fin'ora di un potenziale a "impulso quadro", ovvero un potenziale che sale fino ad un $W$  costante e poi torna a 0 al tempo $\tau$. Se interagiamo con il campo elettromagnetico dobbiamo prendere una interazione con un potenziale oscillante, supponiamo di avere una onda piana
\[
V=W^+ e^{-i\omega t} + W^- e^{i\omega t} 
.\] 
Se la perturbazione è di questo tipo possiamo riprendere tutti i passaggi fatti in precedenza e aggiungere all'esponenziale complessa che avevamo prima il pezzo oscillante del campo EM:
\[
    e^{i\omega_{mn}t} \to e^{i(\omega_{mn}-\omega)t }
.\] 
Quindi possiamo utilizzare il caso costante anche per descrivere la situazione oscillante a patto di inserire questo termine.
\subsubsection{Assorbimento}%
Prendiamo la parte oscillante $W^+e^{-i\omega t}$ ed introduciamola nell'ampiezza $a_{m\to n}$:
\[
    a_{m\to n}^{(1)}(t) =
    \frac{1}{i\hbar }\bra{n}W^+ \ket{m} \frac{e^{i(\omega_{nm}-\omega )t}-1}{i\left( \omega_{mn}-\omega\right)}
.\] 
Quindi si ha che:
\[
    W^{(1) }_{m\to n}(\tau) =
    \left|a^{(1)}_{m\to n}(\tau) \right|^2 =
    \frac{2}{\hbar }\left|\bra{n}W^+ \ket{m}\right|^2 F(\tau,E_n-E_m - \hbar \omega) 
.\] 
Nelle ipotesi della regola d'oro di fermi abbiamo che:
\[
    \overline{P}^{(1)}_{m\to n} = \frac{2\pi}{\hbar }\left|\bra{n}W^+ \ket{m}\right|^2\rho(E_n=E_m+\hbar \omega) 
.\] 
Oppure possiamo scrivere senza integrare:
\[
    P^{(1)}_{m\to n} = \frac{2\pi}{\hbar }\left|\bra{n}W^+ \ket{m}\right|^2\delta (E_n-E_m-\hbar \omega) 
.\] 
\begin{figure}[H]
    \centering
    \incfig{processo-di-assorbimento}
    \caption{Processo di assorbimento}
    \label{fig:processo-di-assorbimento}
\end{figure}
\noindent
\subsubsection{Emissione stimolata}%
Prendendo adesso invece il termine $W^-e^{i\omega t}$ si ottiene con passaggi analoghi:
\[
    \overline{P}^{(1)}_{m\to n} = \frac{2\pi}{\hbar }\left|\bra{n}W^- \ket{m}\right|^2\rho(E_n=E_m-\hbar \omega) 
.\] 
\begin{figure}[H]
    \centering
    \incfig{processo-di-emissione}
    \caption{Processo di emissione stimolata}
    \label{fig:processo-di-emissione}
\end{figure}
\noindent
Quindi i due termini ci descrivono due processi diversi: uno ci fa guadagnare energia e l'altro ce la fa perdere (questo indicano i - e il +). Quindi questi rappresentano il processo di emissione stimolata ed assorbimento, inoltre i due termini $W^+$ e $W^-$ classicamente sono identici. Abbiamo quindi una simmetria completa tra i processi di emissione stimolata ed assorbimento, questa simmetria crea dei problemi a livello termodinamico che verrà risolta con l'aggiunta della emissione spontanea.\\
Vediamo che questi processi avvengono a step di $\hbar \omega$, quantizzando il campo EM questa sarà proprio l'energia del nostro quanto.
\begin{fact}[Interazione con il campo elettromagnetico]{fact:Interazione con il campo elettromagnetico}
L'interazione con il campo EM di un sistema materiale nelle ipotesi della regola d'oro di Fermi avviene tramite cessione o assorbimento di quanti di energia $\hbar \omega$ dal oppure al campo.
\end{fact}
Studiamo adesso il caso in cui gli stati finali ed iniziali hanno una propria distribuzione, in questo caso si utilizza una densità di stati congiunta o ridotta, creata pensando alla simmetria tra emissione ed assorbimento:
\[
    \rho_\text{congiunta} (E_n-E_m) 
.\] 
Che ci dice il numero di stati separati da una energia $E_n-E_m$. Possiamo prendere come esempio il caso di semiconduttori, in tal caso abbiamo abbiamo un gap tra due bande (di conduzione e valenza) paraboliche:
\begin{figure}[H]
    \centering
    \incfig{densità-di-stati-congiunta-per-i-semiconduttori}
    \caption{Densità di stati congiunta per i semiconduttori}
    \label{fig:densità-di-stati-congiunta-per-i-semiconduttori}
\end{figure}
\noindent
Essendo la differenza di due parabole una parabola abbiamo che anche la $\rho$ congiunta preserva la forma funzionale delle due bande in questione.
In questo modo possiamo anche riscrivere la probabilità di transizione:
\[
    \overline{P}^{(1)}_{m\to n} = \frac{2\pi}{\hbar }\left|\bra{n}W^+ \ket{m}\right|^2\rho(E_n-E_m=\hbar \omega) 
.\] 
\subsubsection{Validità dello sviluppo perturbativo}%
Dobbiamo adesso capire quando è possibile fermarci allo sviluppo al primo ordine, sarà necessario che:
\[
    W^{(1) }_{m\to n}(\tau) \ll 1
.\] 
Questo significa che:
\[
\frac{1}{\hbar }\left|\bra{n}W^{\pm} \ket{m}\right|^2\tau \ll 1
.\] 
L'elemento di matrice sarà proporzionale al fattore $\v{E}\cdot \v{d}$ quindi sarà necessario considerare tempi di interazione lunghi rispetto alla larghezza di distribuzione dei livelli ma devono essere corti rispetto al reciproco della intensità del campo.\\
In parole povere dobbiamo considerare campi piccoli:
\[
     \frac{E^2}{\hbar}\tau  \ll 1
.\] 
Vale quindi la seguente disuguaglianza sui tempi di interazione per poter applicare lo sviluppo perturbativo al primo ordine e la regola d'oro di Fermi:
\begin{fact}[Intervallo temporale della perturb. per la regola 'd'oro di Fermi]{fact:Intervallo temporale della perturb. per la regola 'd'oro di Fermi}
\[
    \frac{2\pi}{\Delta\omega}\ll \tau\ll \frac{\hbar }{\left|\v{E}\right|^2}
.\] 
\end{fact}
\subsubsection{Sviluppo perturbativo agli ordini successivi}%
Vediamo che succede andando agli ordini successivi, prendiamo $a^{(1)}(t)$ e le rimettiamo nella parte destra della equazione differenziale e troviamo com'è fatta la probabilità di transizione. In questo caso senza effettuare i conti vediamo il risultato:
\[
    a^{(2)}_{m\to n}(t) =
    \frac{1}{\hbar ^2} \sum_{n}^{} 
    \frac{\bra{k}W^{\pm}\ket{n} \bra{n}W^{\pm} \ket{m}}{\omega_{nm}\pm \omega}
    \cdot 
    \left\{
	\frac{e^{i(\omega_{km}\pm_2\omega)t}-1}{\omega_{km}\pm_2\omega}
    - \frac{e^{i(\omega_{km}\pm \omega)t}-1}{\omega_{kn}\pm\omega}\right\}
.\] 
La cosa importante è il fatto che abbiamo una somma su tutti gli n. Inoltre la nostra $F$ adesso è piccata nel punto in cui la differenza dei due stati è il doppio di $\omega$ del campo magnetico, quindi una volta passato al secondo ordine ho dei processi in cui posso scambiare nergia con il campo magnetico assorbendo o emettendo 2 quanti di energia (o 2 $\hbar \omega$ ), per questo prendono il nome di processi a due fotoni.\\
Questi argomenti agli ordini superiori sono argomenti di ottica non lineare (il numero di fotoni coinvolti è maggiore di 1).\\
Abbiamo fin'ora assunto una onda elettromagnetica monocromatica, vedremo cosa succede quando abbiamo invece una distribuzione di campo elettromagnetico (che è quello che succede tipicamente). In generale avremo una densità di energia $u(\nu)d\nu$, in questo caso dovremmo fare la convoluzione della probabilità monocromatica che abbiamo calcolato con tale densità di energia.
\subsection{Coefficienti di Einstein.}%
\label{sub:Coefficienti di Einstein.}
Vediamo adesso come può essere scritto l'elemento di matrice, per far questo possiamo metterci ad esempio nella Gauge di Coulomb, sostituiamo all'impulso:
\[
    \v{P} \to 
\v{P}- \frac{e}{n}\v{A}\cdot \v{P}
.\] 
In tale Gauge abbiamo che il potenziale di interazione diventa:
\[
    \hat{V}(t) = -\frac{e}{mc}\v{A}\cdot \v{P} + \cancel{\frac{e^2}{2mc^2}\v{A}^2}
.\] 
Dove non consideriamo il termine al secondo ordine. Il caso più interessante o semplice che possiamo trattare è quello in cui $\v{A}$ è una onda piana:
\[
    \v{A}=A_0 \hat{e}\cos(\v{k}\cdot \v{r}-\omega t) 
.\] 
Nel vuoto inoltre il campo elettrico può essere scritto come:
\[
\v{E}= -\frac{1}{c}\frac{\text{d} \v{A}}{\text{d} t} 
.\] 
Quindi possiamo riscrivere il $\hat{V}(t) $ senza esplicitare i conti come 
\[
    \hat{V}(t) = \frac{eE_0}{m\omega} \frac{1}{2}\left(e^{i(\v{k}\cdot \v{r}-\omega t)} + e^{-i(\v{k}\cdot \v{r}-\omega t) }\right) \hat{e}\cdot \v{P}
.\] 
Dove il termine $E_0$  è legato al potenziale vettore tramite:
\[
E_0=-A_0 \frac{\omega}{c}
.\] 
Quindi possiamo scrivere la probabilità con l'elemento di matrice esplicitato:
\[
    \overline{P}^{(1)}_{m\to n}=
    \frac{2\pi}{\hbar }\frac{e^2}{m^2\omega^2}E_0^2 \frac{1}{4}
    \left|
    \bra{\overline{n}}
    e^{\pm i \v{k}\cdot \v{r}}\hat{e}\cdot \v{P} 
    \ket{\overline{m}}
    \right|^2\rho(E_n = E_m \pm \hbar \omega) 
.\] 
\subsubsection{Approssimazione di Dipolo}%
Ci mettiamo adesso in una situazione in cui la lunghezza d'onda della radiazione è maggiore della estensione della funzione d'onda dei livelli che consideriamo (o anche del parametro della funzione d'onda che descrive i nostri livelli) :
\[
\lambda  \gg a
.\] 
ovvero $ka \ll 1$, questa è l'approssimazione di dipolo. \\
Tale approssimazione funziona bene per gli atomi e anche per i solidi. Infatti se $a$ è il passo reticolare ($10^{-10}$) e consideriamo onde nel visibile ($\lambda\sim \mu$m) si ha la disuguaglianza.\\
In questo modo si arriva alla espressione più classica per la regola d'oro di fermi per le transizioni nel campo elettromagnetico:
\begin{fact}[Regola d'oro di Fermi per il campo elettromagnetico]{fact:Regola d'oro di Fermi per il campo elettromagnetico}
\[
    P^{(1) }_{m\to n}= 
    \frac{2\pi}{\hbar }\frac{e^2}{m^2\omega^2}\frac{E_0^2}{4}
    \left|\bra{n}\hat{e}\cdot \v{P} \ket{m}\right|^2
    \rho(E_n=E_m \pm \hbar \omega) 
.\] 
\end{fact}
Tipicamente l'elemento di matrice dato dal dipolo lo possiamo riscrivere sfruttando il commutatore tra $\v{r}$ e la tipica Hamiltoniana $H_0$:
\[
    H_0 = \frac{\left|\v{P}\right|^2}{2m} + U(\v{r}) 
.\] 
Il commutatore in questione si ricava essere:
\[
    \left[\v{r},H_0\right]=i \hbar \frac{\v{P}}{m}
.\] 
Al posto di $\v{P}$ mettiamo il commutare, in questo modo l'elemento di matrice diventa:
\[
    P^{(1) }_{m\to n}= 
    \frac{2\pi}{\hbar }\frac{E_0^2}{4}
    \left|\bra{n}\hat{e}\cdot \v{d} \ket{m}\right|^2
    \rho(E_n=E_m \pm \hbar \omega) 
.\] 
Possiamo adesso trattare un campo elettromagnetico non monocromatico, consideriamo la densità di energia:
\[
    u(\nu) d\nu
    = \underbrace{\frac{\epsilon E_0^2}{2}}_{\text{MKSA}} 
    = \underbrace{\frac{E_0^2}{8\pi}}_{\text{CGS}} 
    = n_\text{fot}\hbar \omega 
.\] 
Con $n_\text{fot} $ che è il numero di fotoni per unità di volume.\\
Possiamo adesso riscrivere la densità di stati congiunta per unità di frequenza:
\[
    \rho(E_{mn}) = \frac{1}{2\pi\hbar }g(\nu_{nm}) 
.\] 
Abbiamo semplicemente cambiato variabile per passare in unità di frequenze. La probabilità non integrata di andare dallo stato m a quello n diventa:
\[
    P^{(1)}_{m\to n} =
    \frac{2\pi}{\hbar ^2}\left|\bra{n}\hat{e}\cdot \v{d} \ket{m}\right|^2
    \delta (\nu_{nm}\pm \nu) u(\nu) d\nu g(\nu_{nm}) 
.\] 
Nel caso in cui interagiamo con radiazione non polarizzata dobbiamo mediare il termine nel Bracket sulle orientazioni del dipolo (visto che viene fuori un $\cos^2$ alla fine della media resterà un fattore $1 /3$), quindi:
\[
    P^{(1)}_{m\to n} =
    \frac{2\pi}{3\hbar ^2}\left|\bra{n}d \ket{m}\right|^2
    \delta (\nu_{nm}\pm \nu) u(\nu) d\nu g(\nu_{nm}) 
.\]
Possiamo adesso integrare in $\nu$ per trovare la probabilità di transizione totale per un campo non polarizzato:
\[
    P^{(1) }_{m\to n}= \frac{2\pi}{3\hbar ^2}\left|\bra{n}d \ket{m}\right|^2
    u(\nu_{nm}) g(\nu_{nm}) 
.\] 
\subsubsection{Coefficiente di emissione $B_{m\to n}$}%
Consideriamo adesso un campo avente $u(\nu_{nm})$ molto più larga di $g(\nu_{nm})$, questo significa assumere la distribuzione dei livelli molto più stretta della banda di frequenza della nostra radiazione (il contrario della monocromaticità).\\
Se integriamo in $\nu_{nm}$ con queste ipotesi la $u(\nu_{nm})$ esce dall'integrale, rimane:
\[
    \overline{P}^{(1)}_{m\to n} =
    u(\nu_{nm}) B_{m\to n}
.\]
In cui il coefficiente $B$ vale:
\[
B_{m\to n} =
\frac{2\pi}{3\hbar ^2}\left|\bra{n}d \ket{m}\right|^2
.\] 
Che è un coefficiente introdotto da Einstein, introdotto assumendo che la probabilità di assorbimento e di emissione stimolata sia proporzionale alla densità di energia del campo elettromagnetico presente. \\
Possiamo notare che questa teoria è valida nel limite perturbativo con la regola d'oro di fermi e con un campo elettromagnetico molto largo rispetto alla distribuzione di energia dei livelli energetici (valida per campi elettromagnetici a banda larga come il corpo nero).\\
Ovviamente i coefficienti $B_{m\to n}$ sono simmetrici, Einstein trovò questa simmetria con un sistema a due livelli all'equilibrio termodinamico con la radiazione, egli impose che tale sistema soddisfacesse la condizione di Boltzmann. In questo modo introdusse anche un coefficiente $A$ di emissione spontanea:
\[
P_{n\to m} = B_{n\to m} u + A
.\] 
Tale nuovo termine è indipendente dal campo elettromagnetico.\\
\subsubsection{Utilizzo della regola d'oro per sistemi a due livelli}%
Dalla forma della equazione per la probabilità possiamo notare una simmetria tra la densità del campo elettromagnetico $u$ e la $g$ che è la densità di stati congiunta del sistema materiale, questo ci dice che la regola d'oro di fermi vale anche se abbiamo due stati ben definiti (su un singolo atomo) purché $\tau$ sia molto maggiore della larghezza di riga del campo elettromagnetico.\\
L'analogia nel dettaglio che facciamo è:
\begin{itemize}
    \item Prima: Distribuzione continua di stati con $\Delta\omega_{n}$ e onda monocromatica $\omega$ ben definita. L'ipotesi di lavoro era $\tau  \gg 2\pi  /\Delta\omega$.
    \item Ora: Stati con energia ben precisa ($E_1$ ed $E_2$) e campo elettromagnetico per niente monocromatico (con $\Delta\omega$). L'ipotesi di lavoro è $\tau  \gg 2\pi  /\Delta\omega$.
\end{itemize}
\subsection{Oltre il metodo perturbativo per il campo EM}%
Possono anche esserci dei casi in cui la trattazione perturbativa non funziona a causa della ampiezza del campo magnetico si avrà quindi che:
\[
    W^{(1)}\sim  1 
.\] 
Questo significa che $\left|E\right|^2 /\hbar \tau  \cancel{\ll }1$. \\
Vediamo in un sistema semplice cosa succede se non possiamo trattare il campo magnetico perturbativamente:
\begin{figure}[H]
    \centering
    \incfig{livelliarimondo}
    \caption{Livelli energetici del sistema. In questo caso il livello 2 è il fondamentale.}
    \label{fig:livelliarimondo}
\end{figure}
\noindent
Incidiamo con una onda elettromagnetica di ampiezza $E_0$ e frequenza $\omega$ mentre i due livelli sono separati da una energia $\hbar \omega_{12}$. La perturbazione $\hat{V}(t) $ sarà:
\[
    \hat{V}(t) = \frac{e}{m\omega} \frac{E_0}{2}\left(\hat{e}\cdot \v{p}\right)
    \left(e^{-i\omega t}+e^{i\omega t}\right)
.\] 
Gli elementi di matrice:
\[
\bra{1}W^+ \ket{2}= \frac{e}{m\omega}\frac{E_0}{2}\bra{1}\hat{e}\cdot \v{p}\ket{2}=
i \frac{E_0}{2}\hat{e}\cdot \v{d}_{12} 
.\] 
Nella quale $\v{d}_{12} = \bra{1}\v{d} \ket{2}$. L'altro elemento invece:
\[
\bra{2}W^- \ket{1}=-
i \frac{E_0}{2}\hat{e}\cdot \v{d}_{21} 
.\] 
Assumiamo inoltre che $\v{d}$ sia reale, quindi abbiamo la simmetria: $\v{d}_{12}=\v{d}_{21}$.\\
Senza fare calcolo perturbativo cerchiamo di risolvere l'equazione di Schrodinger a due livelli sostituendovi la seguente $\psi$.
\[
    \ket{\psi(t) }=
    a_1(t) e^{-iE_1t /\hbar }\ket{1}+a_2(t) e^{-i E_2t /\hbar }\ket{2}
.\] 
Sostituiremo questa funzione d'onda all'interno della equazione di Schrodinger tenendo conto della forma degli elementi di matrice che ci siao scritti, introdurremmo in questo modo l'approssimazione di onda rotante.
