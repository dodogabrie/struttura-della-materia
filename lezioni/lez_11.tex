\lez{11}{13-03-2020}{}
\subsection{Condensazione di Bose-Einsein}%
Abbiamo nelle scorse lezioni ampiamente discusso la distribuzione di Fermi-Dirac, vediamo oggi invece alcune interessanti proprietà di quella di Bose-Einstein.
\[
	\overline{n}_{q} = \frac{1}{\exp\left( \frac{\mathcal{E} - \mu }{kT}\right)-1 }
.\]
Ricordiamo che per un gas di Bosoni il potenziale chimico $\mu $ deve essere necessariamente negativo e che l'occupazione degli stati non è necessariamente inferiore di 1 come invece avviene nel caso della Fermi-Dirac. Per questa distribuzione è possibile avere $\overline{n}_{q} \gg 1$, vedremo che questo ci porterà al condensato di Bose-Einstein.\\
\paragraph{Densità critica e temperatura critica.}
Quando aumenta $\overline{n}_q$ il potenziale chimico si muove verso l'origine "shiftando" l'intera curva di $\overline{n}_{q}$ verso destra: il sistema si popola fino a che $\mu$ non raggiunge il suo valore massimo ($\mu =0$). \\
Essendo il potenziale chimico una funzione monotona del numero di particelle ci aspettiamo che $\mu \to 0$ quando $N\to \infty$. Calcolando il numero di particelle $N$ per il sistema di bosoni ci si rende conto che tale intuizione non risulta rispettata:
\[
	N = \frac{4\pi V g \sqrt{2} m ^{3 /2}}{h^3}
	\int_{0}^{\infty} \frac{\mathcal{E} ^{1 /2}d\mathcal{E} }
	{\exp\left( \frac{\mathcal{E} -\mu }{kT} \right) -1} 
.\] 
Vediamo questo come diventa con $\mu = 0$, dovrebbe divergere $N$ per avere il risultato sperato, invece \footnote{Cambio variabile nell'integrale: $x = \frac{\mathcal{E} }{kT}$}:
\[
	\lim_{\mu \to 0} N = N_{\text{max}}= \frac{4\pi V g \sqrt{2} m ^{3 /2}}{h^3} \left( kT \right)^{3 /2} \int_{0}^{\infty} \frac{x^{1 /2}dx}{e^{x}-1} \label{eq:N-critico}
.\] 
Sfortunatamente l'integrale non diverge, esso fa: $ \sqrt{\pi} /2 \cdot 2.612$. Possiamo allora definire una densità critica
\[\begin{aligned}
    \left( \frac{N}{V} \right)_{\text{c}} &= \left(\frac{\sqrt{2\pi m kT}}{h}\right)^{3} 2.612 =\\ 
					  &=\frac{2.612}{\Lambda ^3}
.\end{aligned}\]
Viceversa possiamo tenere la densità $N /V$ costante ed abbassare la temperatura, in questo modo scopriamo che c'è anche una temperatura critica:
\[
	T_{c} = \frac{1}{2.31 m k} \left( \frac{N}{V} \right)^{2 /3} \frac{\left( 2\pi \hbar  \right) ^2}{\left( 4\pi \sqrt{2}  \right) ^{2 /3}}
.\] 
Dove abbiamo semplicemente invertito la precedente formula.
Sembra quindi che vi sua un limite nella quantità di particelle che posso inserire nel volume disponibile e nella temperatura minima raggiungibile per il gas di Bosoni. \\
Questa stranezza è dovuta alla condensazione di Bose-Einstein, per comprenderne le caratteristiche concentriamoci da prima sulla condensazione classica. 
Per un gas classico vale sempre:
\[
	N = \frac{PV}{kT}
.\] Se aumentiamo la densità del gas a temperatura costante andiamo incontro alla condensazione. Tale passaggio comporta lo stabilizzarsi della pressione sul valore di tensione di vapore: più particelle inserisco aumentando la densità e più particelle entrano nella fase liquida.
\begin{figure}[H]
    \centering
    \incfig{pressione-di-vapore}
    \caption{Pressione di vapore: curve isoterme.}
    \label{fig:pressione-di-vapore}
\end{figure}
\noindent
Per il gas di bosoni il ruolo della tensione di vapore è giocato dall'integrale nella \ref{eq:N-critico}:
\[
	y =\int_{0}^{\infty} \frac{\mathcal{E} ^{\frac{1}{2}d\mathcal{E} }}{\exp\left( \frac{\mathcal{E} -\mu }{kT} \right)-1 } 
.\] 
Ha esattamente lo stesso andamento della pressione del gas classico al diminuire di $\mu$: anche il gas di bosoni cambia fase quando $\mu \to 0$.\\
Quindi le particelle mancanti nel conto devono aver formato una nuova fase, tuttavia stiamo trattando un gas perfetto (particelle non interagenti). Come è possibile transire di fase? Dobbiamo trovare l'errore nella nostra trattazione.\\
Il calcolo di $N$ risulta effettuato nel modo corretto:
\[
    N = \sum_{q}^{} \overline{n}_{q} \to \int_{0}^{\infty} \rho (\mathcal{E}) \overline{n}_q d\mathcal{E}
.\] 
Per poter effettuare tale passaggio al continuo è necessario che:
\[
	kT \gg \frac{\hbar^2}{2m}\left( \frac{2\pi}{L} \right) ^2
.\]
Vogliamo che questa sia rispettata nel momento in cui iniziano a succedere cose strane al nostro gas, ovvero quando $T = T_{C}$, sostituendo questo valore:
\[
	N^{2 /3} \gg \frac{2.31 \left( 4\pi\sqrt{2}  \right)^{2 /3}}{2} \sim 10^0
.\] 
Visto che tale disuguaglianza è rispettata vediamo com'è fatta la $\rho ( \mathcal{E} ) $, sappiamo che in un caso tridimensionale si ha:
\[
	\rho ( \mathcal{E} ) \propto \mathcal{E} ^{1 /2}
.\] 
In particolare $\rho \to 0$ con $\mathcal{E} \to 0$, abbiamo trascurato lo stato fondamentale nel passaggio al continuo!\\
Ricordiamo che classicamente questo non sarebbe un problema: per quanto il fondamentale possa essere popolato è soltanto uno stato a fronte di un insieme continuo di stati (ragionamento effettuato per il passaggio al continuo). \\
Per un gas di Bosoni quando $\mu \to 0$ si ha $\overline{n}_{q}( \mathcal{E} = 0)  \to \infty$ quindi il fondamentale diventa tutt'altro che trascurabile.\\
Con il conto fatto nella \ref{eq:N-critico} ci stiamo perdendo il numero di particelle che finiscono nello stato fondamentale $N_0$, il numero di particelle totali sarà dato dalla somma di quello che abbiamo calcolato prima $N^{*}$ con questo nuovo termine.
\[
	N = N_0 + N^{*}
.\] 
con $N^{*}$ che ricordiamo essere sempre quello di prima:
\[
	N^{*} = \int_{0}^{\infty} 
	\frac{\rho( \mathcal{E} ) d\mathcal{E} }
	{\exp\left( \frac{\mathcal{E} }{kT} \right) -1} 
.\] 
Se prendiamo la funzione di distribuzione, vi sostituiamo l'energia del fondamentale e sviluppiamo attorno a $\mu = 0$ abbiamo che:
\[
	\overline{n}_{0} = \frac{1}
		{\exp\left( -\frac{\mu }{kT} \right) -1 } \approx - \frac{kT}{\mu }
.\] 
Abbiamo quello che di si aspettava: il numero di particelle nel fondamentale effettivamente diverge. L'illusione di avere una densità critica era dovuta al fatto che le particelle vanno tutte in questo stato che avevamo considerato trascurabile essendo soltanto uno. \\ 
Questo fenomeno prende il nome di condensazione di Bose-Einstein. 
Non è una comune condensazione, infatti non possiamo avere la conferma visiva di un passaggio di stato, come invece si ha nel passaggio di stato da gassoso a liquido. 
Questa è una condensazione nello spazio delle fasi \footnote{o nello spazio degli impulsi} per le particelle con $p =0$. \\
Il conto che abbiamo fatto per trovare il numero critico di particelle nella \ref{eq:N-critico} resta vero, soltanto che quel numero rappresenta il numero di particelle che sono nell'eccitato quando $T< T_{c}$ (quindi quando $\mu = 0$ ). Possiamo assumere che alla temperatura critica, quindi quando la condensazione inizia, che:
\[
	N^{*}( T_{\text{crit}})  \approx N
.\] 
Vista l'espressione della $N^{*}$ che è funzione di $T^{3 /2}$ si dovrà avere la seguente espressione quando la temperatura scende al di sotto di quella critica:
\begin{align}
	&N^{*} = N \left( \frac{T}{T_{c}} \right)^{3 /2}& & \text{con } T<T_{c}
.\end{align}
Se approssimiamo inoltre le particelle divise tra quelle che non stanno nel fondamenta e e quelle che vi stanno abbiamo anche che $N = N_0 + N^{*}$, di conseguenza:
\[
	N_0 = N\left( 1 -\left( \frac{T}{T_{c}} \right)^{3 /2}
 \right) 
.\] 
Sopra $T>T_{c}$ si ha invece $N_0 \sim 0$. Ricordiamo che resta comunque la disuguaglianza $\overline{n_0}>\overline{n_{q}}$, il motivo per cui nonostante quest'ultima si ha $N_0 \sim 0$ è che il fondamentale è uno soltanto a fronte dei tantissimi stati $q_{i}$ del gas, che quindi contengono la quasi totalità delle particelle. Di seguito di riportano i plot delle due popolazioni discusse
\begin{figure}[H]
    \centering
    \incfig{particelle-nel-fondamentale-e-particelle-negli-stati-eccitati-in-funzione-della-temperatura}
    \caption{Particelle nel fondamentale e particelle negli stati eccitati in funzione della temperatura}
    \label{fig:particelle-nel-fondamentale-e-particelle-negli-stati-eccitati-in-funzione-della-temperatura}
\end{figure}
\noindent
Per lo stesso motivo per cui in realtà abbiamo 
\[
	N = N_0 + N^{*}= N_0 + 
	\int_{0}^{\infty} 
	\rho ( \mathcal{E} )\overline{n}_{q}( \mathcal{E} ) 
	d\mathcal{E} 
\]
non è corretto dire che $\mu $ sia esattamente zero sotto a $T_{c}$, è zero soltanto nell'ipotesi in cui possiamo trascurare il valore di $N_0$ alla temperatura critica:
\[
	N_0( T_{c}) \approx 0 \implies \mu \approx 0
.\] 
\paragraph{Energia}
La formula per l'energia a noi nota rimane valida anche in questo caso (notiamo che si ha $\mu=0$):
\[
	E = \int_{0}^{\infty} 
	\frac{\mathcal{E} \rho ( \mathcal{E} ) d\mathcal{E} }
	{\exp\left( \frac{\mathcal{E} }{kt} \right) -1} 
.\] 
Perché le particelle nello stato fondamentale sono ad energia nulla, cambiando variabile abbiamo la seguente espressione:
\[
	E = \frac{4\pi V g \sqrt{2} }
	{h^3}\left( m k T \right) ^{3 /2} kT 
	\int_{0}^{\infty} \frac{x^{3 /2}}{e^{x}-1}dx  
.\] 
Quest'ultimo integrale converge numericamente, togliendo di mezzo qualche costante si arriva alla forma:
\[
	E = 0.770 NkT \left( \frac{T}{T_{c}} \right) ^{3 /2} \propto T^{5 /2}
.\] 
\paragraph{Calore specifico}
Nota l'energia possiamo anche conoscere l'andamento del calore specifico:
\[
	C_{V}= \left.\frac{\partial E}{\partial T} \right|_{V} = 1.9 Nk \left( \frac{T}{T_{c}} \right) ^{3 /2} \propto T^{3 /2}
.\] 
quindi il calore specifico segue l'andamento di $N^{*}$: il numero di particelle nel livello eccitato. \\
Per $T > T_{c}$ se teniamo il numero di particelle fissato non si può più mettere $\mu = 0$, in particolare dovremo inserire nella formula per l'energia $\mu(T)$, questo avrebbe delle conseguenze anche sulla derivata per il calore specifico. Saltiamo questo conto tedioso e concentriamoci sul caso $T\to \infty$.\\
Senza effettuare conti in questo limite sappiamo che torniamo al caso classico di Boltzmann ($N= 3/2 kT$ ), quindi si ha un calore specifico costante in $T$.
\begin{figure}[H]
    \centering
    \incfig{calore-specifico-al-variare-della-temperatura-per-un-gas-di-bosoni}
    \caption{Calore specifico al variare della temperatura per un gas di bosoni}
    \label{fig:calore-specifico-al-variare-della-temperatura-per-un-gas-di-bosoni}
\end{figure}
\noindent
Il picco del grafico è l'andamento della funzione nella zona che non ci siamo calcolati, anche questo è un andamento tipico del calore specifico in presenza di transizioni di fase.\\
\paragraph{Pressione}
Al di sotto della temperatura critica possiamo calcolarci anche la pressione. È possibile calcolare questa come per tutte le particelle aventi dispersione $p^2 /2m$ (per i bosoni va bene):
 \[
	 P = \frac{2E}{3V} = 1.2 \frac{kT}{\Lambda ^3} \propto \left( kT \right) ^{5 /2} 
.\] 
In particolare questa ci dice che sotto la temperatura critica $T_{c}$ la pressione è indipendente dal volume. Facendo una compressione isoterma del sistema la pressione non cambia.\\ 
Questo avveniva anche per i passaggi di stato da che abbiamo visto nella Sezione \ref{sec:transizioni-di-fase} nell'esempio di Figura \ref{fig:curve-isoterme-pt-pv}. \\
Possiamo anche notare che sotto la temperatura critica $C_{P}$ diverge perchè $\partial P /\partial V$ è nullo.\\
Il grafico $PT$ per questo particolare sistema sarà del seguente tipo:
\begin{figure}[ht]
    \centering
    \incfig{andamento-pt-per-un-gas-di-bosoni}
    \caption{Andamento PT per un gas di Bosoni}
    \label{fig:andamento-pt-per-un-gas-di-bosoni}
\end{figure}
Sappiamo che la condizione di coesistenza delle fasi è che sia costante $\mu ( P,T)$, nel nostro caso tale condizione si ha per $\mu = 0$.\\
Notiamo che la zona di condensato non è raggiungibile se non per $T= 0$, quindi non si avrà mai tutto il sistema nello stato condensato.

\subsection{Oscillatore armonico}%
Vediamo dal punto di vista termodinamico un sistema di oscillatori armonici, partiamo da un caso semplice.
\paragraph{Oscillatore armonico classico}
L'Hamiltoniana per un singolo oscillatore questo caso sarà:
\[
	H( p,q) = \frac{1}{2}m\omega ^2q^2 + \frac{1}{2m}p^2
.\] 
Possiamo scrivere la funzione di partizione dell'oscillatore chiamando $\beta  = 1 /kT$:
\[
	Z_1( \beta )  = \frac{1}{h}\int_{-\infty}^{\infty} \int_{-\infty}^{\infty} \exp\left( -\beta \left( H( q,p)  \right)  \right) dqdp  
.\] 
Essendo l'Hamiltoniana una forma quadratica separabile in $p$ e $q$ possiamo dividere l'integrale nel prodotto di due integrali e svolgerli separatamente.
\[
	Z_1 = \frac{1}{h}\left( \frac{2\pi}{\beta m\omega ^2} \right) ^{1 /2}\left( \frac{2\pi m }{\beta } \right) ^{1/ 2} = \frac{1}{\beta \hbar \omega }= \frac{kT}{\hbar\omega }
.\] 
Se il sistema è composto da $N$ di questi oscillatori indipendenti allora avremo anche che:
\[
	Z = Z_{1}^{N}
.\] 
Notiamo che con questa funzione di partizione stiamo assumendo gli oscillatori come distinguibili. La distinguibilità questa volta è giustificata dal fatto che tratteremo sistemi in cui ogni singolo oscillatore sarà l'equivalente di un livello di energia o di un modo normale di oscillazione. Questi due ultimi saranno sicuramente distinguibili, per i modi normali ad esempio ognuno avrà la sua frequenza.\\
Visto che l'integrale per $Z_1$ è "separabile" possiamo generalizzare ad un caso 3D:
\[
	Z = Z^{3D}= \left( Z^{1D} \right)^3 = Z_1^{3N}
.\] 
Possiamo ricavare l'energia libera:
\[
	F = -kT \ln Z = NkT \ln \left( \frac{\hbar\omega }{kT} \right) 
.\] 
In tre dimensioni si ha semplicemente:
\[
	F = -kT \ln Z = 3NkT \ln \left( \frac{\hbar\omega }{kT} \right) 
.\] 
Possiamo adesso calcolare la pressione, in questo caso non avendo dipendenza dal volume per l'energia libera si avrà $P = 0$. \\
È inoltre possibile calcolare l'entropia:
 \[
	 S = -\left.\frac{\partial F}{\partial T} \right|_{V, N} = Nk \left[ \ln \left( \frac{kT}{\omega \hbar} \right) +1  \right] 
.\] 
L'energia media invece sarà:
\[
	E = F + TS = NkT
.\] 
Il calore specifico:
\[
	C_{V} = \frac{\partial E}{\partial T} = Nk
.\] 
Quindi ogni oscillatore (termine cinetico e termine potenziale) porta con se un contributo all'energia di $kT$.
\begin{defn}[Teorema di equipartizione dell'energia]{def:Teorema di equipartizione dell'energia}
	Ogni termine quadratico dell'Hamiltoniana porta un contributo all'energia media di $kT /2$.
\end{defn}
Nel caso dell'oscillatore armonico abbiamo infatti sei termini quadratici nella Hamiltoniana, conforme con il teorema. Tale teorema è un teorema che funziona in ambito classico, non quantistico.
\paragraph{Oscillatore armonico quantistico}
Per questo caso abbiamo dei livelli energetici che sono quantizzati
\[
	\mathcal{E} _{n}=\left( n + \frac{1}{2} \right) \hbar \omega 
.\] 
È necessario quindi rifare il conto di $Z_1$ inserendo questi livelli energetici
\[
	Z_1 = \sum_{n=0}^{\infty} \exp\left[ -\beta \left( n + \frac{1}{2} \right) \hbar\omega  \right] =
	\frac{\exp\left( -\frac{1}{2}\beta \hbar\omega  \right) }{1- \exp\left( -\beta \hbar\omega  \right) }
.\] 
Qundi la funzione di partizione totale sarà:
\[
	Z = Z_1^{N} = \frac{\exp\left( \frac{N}{2}\beta \hbar\omega  \right) }{\left[ 1-\exp\left( -\beta \hbar\omega  \right)  \right] ^N}
.\] 
Con gli stessi passaggi di prima troviamo che:
\[
    	F = NkT\ln\left[ 2\sinh\left( \frac{\beta \hbar\omega }{2} \right)  \right] = N \left[ \frac{1}{2} \hbar\omega + kT \ln\left( 1-\exp\left( -\beta \hbar\omega  \right)  \right)  \right] 
.\] 
Anche nel caso quantistico abbiamo $P=0$ come nel caso classico. L'entropia invece sarà:
\[
	S = Nk\left[ \frac{\beta \hbar\omega }{\exp\left( -\beta \hbar\omega  \right) -1} - \ln\left( 1-\exp\left( -\beta \hbar\omega  \right)  \right)  \right] 
.\] 
Quindi l'energia
\[
	E = N\left[ \frac{1}{2}\hbar\omega + \frac{\hbar\omega }{\exp\left( \frac{\hbar\omega }{kT} \right) -1} \right] =
	N\hbar\omega \left[ \frac{1}{2}+ \frac{1}{\exp\left( \frac{\hbar\omega }{kT} \right) -1} \right]
.\] 
Questa è l'energia media di $N$ oscillatori armonici aventi tutti l'energia $\hbar\omega $. È interessante confrontare questa con i livelli energetici del singolo oscillatore.
Nell'espressione scritta sopra per l'energia di $N$ oscillatori abbiamo che l'energia è definita come una quantità moltiplicata per $N$, è ragionevole pensare quindi che l'energia del singolo oscillatore sarà proprio quella quantità:
\[
	\mathcal{E} = \hbar\omega \left( \frac{1}{2} + \frac{1}{\exp\left( \frac{\hbar\omega }{kT} \right) -1} \right) 
.\] 
Quindi possiamo interpretare il secondo termine tra parantesi come il livello di oscillazione medio per fare una analogia con l'oscillatore singolo:
\[
	\overline{n} = \frac{1}{\exp\left( \frac{\hbar\omega }{kT} \right) -1}
.\] 
Tuttavia possiamo anche avere una visione completamente diversa: $\overline{n}$ come il numero medio di "quasi particelle" che si trovano nello stato con energia $\hbar\omega $. \\
Stiamo quindi costruendo una analogia tra l'oscillatore armonico quantistico e la statistica studiata fin'ora definendo il numero medio di particele che in realtà non ci sono.\\
In questa lettura l'oscillatore armonico corrisponde al livello di energia $\hbar\omega $ popolato da un numero medio di particelle $\overline{n}$. Queste quasi particelle porteranno le informazioni fisiche per descrivere lo stato di eccitazione dell'oscillatore stesso. \\
I quanti discussi hanno un popolamento medio che segue la Bose-Einstein, quindi sono Bosoni un pò particolari: non hanno potenziale chimico\footnote{Sono descritti da una Bose-Einstein che non contiene la $\mu $.}.\\
Non avere il potenziale chimico indica che non c'è una legge di conservazione del numero di particelle globali.\\
Si ha infatti che il potenziale chimico può essere definito come il moltiplicatore di Lagrange indotto dal vincolo di avere un numero di particelle costanti nell'universo. Non avere il potenziale significa far cadere tale vincolo (Patrhia).
Essendo quanti di eccitazione è lecito far cadere la conservazione del numero di particelle.\\
L'energia di punto zero:
\[
	E_0 = \frac{N}{2}\hbar\omega 
.\] 
é importante perché fa andare l'energia media al limite asintotico fisicamente corretto tornando nel limite classico. 
Tuttavia tale energia non è spesso accessibile dalle misure, il caso tipico in cui lo si capisce sarà la descrizione fisica del campo elettromagnetico: le misure su questo campo riescono a campionare soltanto l'eccitazione, non si riesce ad avere accesso all'energia di punto zero con le misure pratiche.\\
Questo risolve il problema del fatto che l'energia del punto zero nel campo elettromagnetico diverge. Vuol dire che probabilmente c'è nuova fisica laggiù?

